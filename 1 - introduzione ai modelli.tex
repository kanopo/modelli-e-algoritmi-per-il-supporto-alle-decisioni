\chapter{Introduzione ai modelli}
\section{Tipologie di algoritmi}

\begin{enumerate}
  \item Algoritmi costruttivi: Gli algoritmi costruttivi vengono utilizzati per costruire una soluzione in modo incrementale, aggiungendo passo dopo passo componenti o elementi al problema. Questi algoritmi partono da una soluzione vuota e aggiungono iterativamente elementi considerando regole o euristiche specifiche. Gli algoritmi costruttivi sono spesso efficienti dal punto di vista computazionale, ma potrebbero non garantire la soluzione ottimale. Tuttavia, possono essere utili quando è richiesta una soluzione veloce o approssimata.
  \item Algoritmi di enumerazione: Gli algoritmi di enumerazione esplorano in modo sistematico tutte le possibili soluzioni di un problema, esaminando ogni possibile combinazione. Questi algoritmi possono essere implementati utilizzando tecniche come l'albero delle decisioni o la generazione di tutte le permutazioni. Gli algoritmi di enumerazione possono garantire la completezza (esaminando tutte le soluzioni) ma possono richiedere un tempo di esecuzione elevato per problemi di dimensioni significative.
  \item Algoritmi di raffinamento locale: Gli algoritmi di raffinamento locale si concentrano sulla ricerca di soluzioni ottimali all'interno di un'area di ricerca limitata del problema. Questi algoritmi partono da una soluzione iniziale e iterativamente esplorano soluzioni vicine, apportando miglioramenti incrementali fino a quando non viene raggiunta una soluzione che soddisfa i criteri di ottimalità specificati. Gli algoritmi di raffinamento locale sono particolarmente efficaci per problemi in cui una piccola modifica nella soluzione può portare a un miglioramento sostanziale.
\end{enumerate}

