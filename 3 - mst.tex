\chapter{Minimum Spanning Tree}

Un Minimum Spanning Tree (MST), tradotto in italiano come Albero di Supporto Minimo, è un albero di supporto di peso minimo in un grafo non diretto e connesso. L'MST è costituito da un sottoinsieme di archi del grafo che connette tutti i nodi in modo tale che il peso totale degli archi sia il più basso possibile.

Le caratteristiche principali degli MST sono:

\begin{enumerate}
    \item \textbf{Connessione}: Un MST deve connettere tutti i nodi del grafo, garantendo che non ci siano nodi isolati.
    \item \textbf{Aciclicità}: Un MST non deve contenere cicli, quindi non può avere archi che creano loop all'interno dell'albero.
    \item \textbf{Peso minimo}: Un MST ha il peso totale degli archi più basso possibile tra tutti gli alberi che soddisfano le prime due caratteristiche.
\end{enumerate}

Gli MST hanno diverse applicazioni pratiche, tra cui:

\begin{itemize}
    \item Reti di comunicazione: Un MST può essere utilizzato per collegare un insieme di punti in una rete di comunicazione minimizzando il costo totale dei collegamenti.
    \item Progettazione di reti stradali: Gli MST possono essere utilizzati per pianificare reti stradali efficienti, dove gli archi rappresentano le strade e il peso degli archi può essere la distanza o il tempo di percorrenza.
    \item Analisi dei dati: Gli MST possono essere utilizzati per individuare le relazioni più rilevanti o significative tra i dati, ad esempio nella visualizzazione delle relazioni tra punti di dati su una mappa.
\end{itemize}


\section{Algoritmo greedy per MST}
L'algoritmo greedy per MST (Minimum Spanning Tree) è un approccio basato sulla selezione di archi in base al loro peso. L'idea principale è quella di selezionare ripetutamente l'arco di peso minimo che collega un nodo dell'MST esistente a un nodo non ancora raggiunto, finché non viene creato un albero che connette tutti i nodi del grafo.



\begin{enumerate}
  
  \item Inizializzazione: Parti da un grafo non diretto e connesso $G = (V, E)$, dove $V$ rappresenta l'insieme dei nodi e $E$ rappresenta l'insieme degli archi. Crea un insieme vuoto MST che conterrà gli archi dell'albero di supporto minimo.
  \item Seleziona un nodo di partenza arbitrario. Questo può essere fatto in modo casuale o seguendo una strategia specifica, ad esempio selezionando il primo nodo dell'insieme dei nodi.
  \item Finché non sono stati raggiunti tutti i nodi:
    \begin{enumerate}
      \item Seleziona l'arco di peso minimo che collega un nodo nell'MST esistente a un nodo non ancora raggiunto. Questo arco deve essere selezionato tra gli archi che collegano il nodo raggiunto all'esterno dell'MST.
      \item Aggiungi l'arco selezionato all'MST.
      \item Marca il nodo raggiunto come "visitato" o "raggiunto".
    \end{enumerate}
  \item Alla fine del processo, l'MST conterrà tutti gli archi necessari per connettere tutti i nodi del grafo in modo che il peso totale dell'MST sia minimo.
\end{enumerate}
L'algoritmo greedy per MST può essere implementato utilizzando diverse strutture dati, come ad esempio una coda di priorità (heap) per selezionare l'arco di peso minimo in modo efficiente.

Ecco un esempio di implementazione dell'algoritmo greedy per MST in Python:

\begin{lstlisting}
  
def greedy_mst(graph):
    mst = []  # Insieme di archi dell'MST
    visited = set()  # Insieme di nodi visitati
    start_node = list(graph.keys())[0]  # Nodo di partenza arbitrario
    
    visited.add(start_node)
    
    while len(visited) < len(graph):
        min_weight = float('inf')
        min_edge = None
        
        # Scansiona gli archi collegati ai nodi visitati
        for node in visited:
            for neighbor, weight in graph[node]:
                if neighbor not in visited and weight < min_weight:
                    min_weight = weight
                    min_edge = (node, neighbor)
        
        if min_edge:
            mst.append(min_edge)
            visited.add(min_edge[1])
    
    return mst
\end{lstlisting}

Questo è solo un esempio di implementazione e può variare a seconda delle specifiche del problema. Assicurati di adattare l'algoritmo in base alle tue esigenze specifiche.

\subsection{Correttezza algoritmo}


La complessità dell'algoritmo greedy per MST dipende dalla rappresentazione del grafo e dalla struttura dati utilizzata.

Assumendo che il grafo sia rappresentato come una lista di adiacenza, con $n$ nodi e $m$ archi, e che si utilizzi una coda di priorità (heap) per selezionare l'arco di peso minimo, la complessità dell'algoritmo è:

\begin{itemize}
    \item Inizializzazione: $O(n)$
    \item Ciclo principale: $O(m \log m)$
\end{itemize}

All'interno del ciclo principale, l'estrazione dell'arco di peso minimo richiede un'operazione di estrazione minima dalla coda di priorità, che ha una complessità di $O(\log m)$. Il ciclo principale viene eseguito $m$ volte.

Quindi, la complessità complessiva dell'algoritmo greedy per MST è $O(n + m \log m)$.

È importante notare che se il grafo viene rappresentato come una matrice di adiacenza invece di una lista di adiacenza, la complessità può essere diversa, ad esempio $O(n^2)$ per la ricerca dell'arco di peso minimo.
\subsection{Complessit\'a dell'algoritmo}






\section{Algoritmo MST-1 - Prim}
(Penso sia l'algoritmo di Prim ma al prof piace inventarsi della roba a caso)

% \begin{enumerate}
%   \item Scegli un vertice arbitrario come vertice di partenza per l'MST.
%   \item Segui i passaggi da 3 a 5 finché ci sono vertici che non sono inclusi nell'MST (chiamati vertici di confine).
%   \item Trova gli archi che collegano qualsiasi vertice dell'albero con i vertici di confine.
%   \item Trova l'arco di peso minimo tra questi archi.
%   \item Aggiungi l'arco scelto all'MST se non forma un ciclo.
%   \item Restituisci l'MST e termina.
% \end{enumerate}
%

\begin{center}
\begin{tikzpicture}[auto,node distance=2.5cm]
    \node (A) {A};
    \node (B) [below left of=A] {B};
    \node (C) [below right of=A] {C};
    \node (D) [below right of=B] {D};
    \node (E) [right of=C] {E};
    \node (F) [below right of=E] {F};
    \node (G) [below left of=F] {G};

    \path (A) edge node {2} (B)
              edge node {5} (C)
          (B) edge node {1} (D)
              edge node {3} (C)
          (C) edge node {2} (D)
              edge node {4} (E)
          (D) edge node {3} (F)
          (E) edge node {1} (F)
          (F) edge node {2} (G);
\end{tikzpicture}\end{center}















%
% \chapter{Minimum Spanning Tree}
%
%
% I problemi di MST(Minimum Spanning Tree o alberi di supporto a peso minimo), dato un grafo non orientato con pesi sugli archi, si vuole 
% determinare tra gli alberi di supporto del grafo quello a peso minimo.
%
% \section{Algoritmo greedy per MST}
%
%
% \begin{enumerate}
%   \item Si ordinano tutti gli archi del grafo in ordine crescente del peso.
%   \item Si pone $E_T = \emptyset$ ($E_T$ è l'insieme degli archi dell'albero di supporto) e $k = 1$
%   \item Se $E_T = V - 1$, STOP e si restituisce l'albero $T = (V, E_T)$ altrimenti proseguo
%   \item Se l'arco $e_k$ non forma cicli con gli archi in $E_T$ lo aggiungo agli archi $E_T$, altrimenti lo scarto
%   \item Pongo $k = k + 1$ e torno al passo 3
% \end{enumerate}
%
%
% \subsection{Correttezza dell'algoritmo}
%
% Non è detto che l'algoritmo greedy restituisca l'MST.
%
% \subsection{Complessità dell'algoritmo}
%
% L'operazione più costosa dell'algoritmo è quella di ordinare gli archi in ordine crescente di peso.
%
% Se siamo in presenza di un grafo denso, la complessità diventa $O(|E| \log(|E|))$, dove $|E|$ è il numero di archi.
%
%
% \section{Vedere su yt gli algoritmi MST1 e MST2}
% perche il prof li spiega leggermente male, come quasi tutto questo corso.
%
%
%
% \section{Shortest path}
%
% Nei problemi di cammino a costo minimo, dadto un grafo $G=(V,A)$ con costo (distanza) $d_{ij}$ per ogni 
% $(i, j) \in A$ e dati due nodi $s, t \in V$ dove i due nodi sono diversi fra di loro, vogliamo individuare un cammino
% elementare orientato da $s$ a $t$ di costo minomo.
%
% \subsection{Algoritmo di Dijkstra}
% Valido solo se $d_{ij} \geq 0 \forall (i, j) \in A$.
% Restituisce i cammini minimi tra un nodo scelto $s \in V$ e tutti gli altri nodi.
%
% \subsection{Algoritmo di Floyd-Warshall}
% Valido anche per distanze negative a patto che non siano presenti cicli a costo negativo(in questo caso restituisce un ciclo a costo negativo).
%
%
