
\chapter{Matching}

\section{Algoritmo}


\textbf{Algoritmo Ungherese (Kuhn-Munkres) per il Matching}

\begin{algorithm}[H]
\SetAlgoLined
\KwIn{Matrice dei costi $C$}
\KwOut{Matching massimo}

Inizializza una matrice ausiliaria $M$ delle dimensioni della matrice dei costi $C$ con tutti gli elementi inizializzati a $0$\;

\While{Non è stato trovato un matching massimo}{
  Trova un cammino aumentante nella matrice dei costi $C$\;
  \If{Il cammino aumentante non esiste}{
    Termina l'algoritmo e restituisci il matching corrente\;
  }
  \Else{
    Aggiorna il matching $M$ tramite il cammino aumentante\;
  }
}

\Return{Matching massimo $M$}\;
\caption{Algoritmo Ungherese (Kuhn-Munkres)}
\end{algorithm}

L'algoritmo ungherese è utilizzato per trovare il matching massimo in un grafo bipartito completo. La matrice dei costi $C$ rappresenta i costi associati agli archi del grafo, dove $C[i][j]$ rappresenta il costo dell'arco tra il nodo $i$ del primo insieme e il nodo $j$ del secondo insieme. L'algoritmo opera sulla matrice ausiliaria $M$, che tiene traccia degli archi selezionati nel matching.

L'algoritmo procede iterativamente, cercando un cammino aumentante nella matrice dei costi. Un cammino aumentante è un cammino alternante che parte da un nodo libero nel primo insieme, passa attraverso un arco non selezionato, e arriva a un nodo libero nel secondo insieme. Se un cammino aumentante viene trovato, l'algoritmo aggiorna il matching selezionando gli archi del cammino aumentante e rimuovendo gli archi alternativi. Se non viene trovato un cammino aumentante, l'algoritmo termina e restituisce il matching massimo corrente.

L'algoritmo ungherese garantisce la ricerca del matching massimo in un grafo bipartito completo con complessità di tempo $O(n^3)$, dove $n$ è la dimensione del grafo.

