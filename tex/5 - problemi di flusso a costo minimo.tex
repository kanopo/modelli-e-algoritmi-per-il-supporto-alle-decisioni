\chapter{Problemi di flusso}
\section{Algoritmo di Ford-Fulkerson}

Per risolvere un problema di flusso, si utilizza l'algoritmo di Ford-Fulkerson, che è uno dei principali algoritmi per trovare il flusso massimo in una rete. Ecco una spiegazione dell'algoritmo di Ford-Fulkerson:

\textbf{Algoritmo di Ford-Fulkerson per il Flusso Massimo}

\begin{algorithm}[H]
\SetAlgoLined
\KwIn{Grafo di rete $G$, nodo di origine $s$, nodo di destinazione $t$}
\KwOut{Flusso massimo nella rete}

Inizializza un flusso $f(u, v) = 0$ per ogni arco $(u, v)$ nel grafo\;

\While{Esiste un cammino $p$ da $s$ a $t$ nel grafo residuo $G_f$}{
  Calcola la capacità residua minima $c_f(p)$ del cammino $p$\;
  Aggiorna il flusso $f(u, v)$ su ogni arco $(u, v)$ del cammino $p$ aggiungendo $c_f(p)$ al flusso\;
  Aggiorna il flusso residuo $f(u, v)$ e il flusso inverso $f(v, u)$ sui rispettivi archi $(u, v)$ e $(v, u)$ nel grafo\;
}

\Return{Il flusso massimo nella rete}\;
\caption{Algoritmo di Ford-Fulkerson}
\end{algorithm}

L'algoritmo di Ford-Fulkerson opera su una rete di flusso, che è rappresentata da un grafo diretto in cui gli archi hanno una capacità massima di flusso. L'algoritmo cerca iterativamente un cammino da $s$ (nodo di origine) a $t$ (nodo di destinazione) nel grafo residuo, che è il grafo originale con i flussi correnti sottratti dalle capacità degli archi. Se viene trovato un cammino, l'algoritmo determina la capacità residua minima lungo quel cammino e incrementa il flusso su ogni arco del cammino in base a questa capacità residua minima. Questo processo continua fino a quando non esistono più cammini da $s$ a $t$ nel grafo residuo.

L'algoritmo di Ford-Fulkerson restituisce il flusso massimo nella rete una volta che non ci sono più cammini da $s$ a $t$ nel grafo residuo. È importante notare che l'algoritmo può richiedere più di una singola esecuzione per ottenere il flusso massimo, poiché può esistere più di un cammino da $s$ a $t$ con capacità residua non utilizzata.

Esistono diverse varianti e miglioramenti dell'algoritmo di Ford-Fulkerson, come l'algoritmo di Edmonds-Karp che utilizza la ricerca in ampiezza per trovare il cammino più corto nel grafo residuo. Altri algoritmi noti per il flusso massimo includono l'algoritmo di Dinic e l'algoritmo di Push-Relabel.




% \chapter{Problemi di flusso}
%
% \section{Classificazione dei nodi}
% \begin{itemize}
%   \item \textbf{Sorgente:} Nodo che ha solo archi uscenti
%   \item \textbf{Destinazione:} Nodo che ha solo archi entranti
%   \item \textbf{Nodo intermedio:} Nodo che ha sia archi entranti che uscenti
% \end{itemize}
%
% \section{Problema di flusso a costo minimo}
% Far arrivare il prodotto dalla sorgente alla destinazione con il costo minimo.
%
% \section{Soluzione di base}
% Si ottiene ponendo a $0$ il flusso di tutti gli archi che fanno parte dell'albero di supporto.
%
% \section{Ammissibilit\`a e degenerazione}
% \begin{itemize}
%   \item \textbf{Ammissibilit\`a:} Se i flussi in base hanno valore \textbf{non negativo}
%   \item \textbf{Degenerazione:} Se i flussi in base hanno valore \textbf{nullo}
% \end{itemize}
%
%
% \section{Coefficiente di costo ridotto}
% Il \textbf{coefficiente di costo ridotto} misura la variazione del valore dell'obiettivo al crescere dell'unit\`a del valore del flusso su un arco fuori base.
%
% \section{Condizione di ottimalit\`a}
% Se i coefficienti di consto ridotto degli archi fuori base sono non negativi, la soluzione di base \`e ottima.
%
%
% \subsection{Calcolo del coefficiente di costo ridotto}
% \begin{enumerate}
%   \item Prendo un arco fuori base
%   \item Aggiungo l'arco all'albero di supporto
%   \item Considero l'unico ciclo che si \`e formato
%   \item Fisso il verso del ciclo come il verso del'arco fuori base che ho aggiunto
%   \item Calcolo il coefficiente di costo ridotto facendo una somma algebrica di tutti gli archi del ciclo, sommando se l'arco che si attraversa ha direzioen concorde con l'arco fuori base inserito, sottraendo altrimenti.
% \end{enumerate}
%
%
% \section{Condizione di illimitatezza}
% Se l'aggiunta di un'arco fuori base crea un ciclo di costo negativo, il problema \`e illimitato.
%
%
% \section{Cambio di base}
% Se le condizioni di ottimalit\`a e illimitatezza non sono soddisfatte, \`e necessario cambiare base.
%
% Si procede togliendo un arco dalla base e aggiungendone un altro.
%
% Per aggiungere un arco in base, bisogna scegliere un arco fuori base che abbia coefficiente di costo ridotto negativo.
%
% \subsection{Aggionamento dei flussi}
% \begin{enumerate}
%   \item Aggiungo l'arco in base
%   \item Si forma il ciclo secondo il verso dell'arco fuori base
%   \item Porto a $\Delta$ il flusso dell'arco fuori base
%   \item Aumento di $$\Delta$$ il flusso degli archi che hanno verso concorde con l'arco fuori base
%   \item Diminuisco di $$\Delta$$ il flusso degli archi che hanno verso opposto all'arco fuori base
% \end{enumerate}
%
%
%
% \section{Algoritmo del simplesso su rete}
% L'algoritmo del simplesso su rete \`e un algoritmo per risolvere il problema di flusso a costo minimo.
%
% Itera il cambio di base finch\`e non trova una soluzione ottima.
%
% \begin{enumerate}
%   \item Verifica di ottimalit\`a
%   \item Verifica di illimitatezza
%   \item Cambio di base
% \end{enumerate}




% \section{Problema di flusso massimo}
%
% Il problema di flusso massimo consiste nel trovare il flusso di massimo valore che può essere trasmesso attraverso una rete di flusso. In altre parole, data una rete di flusso con una sorgente e un pozzo, il problema di flusso massimo cerca di trovare il flusso massimo che può essere inviato dalla sorgente al pozzo attraverso la rete, rispettando i vincoli di capacità degli archi.
%
% Il flusso in una rete di flusso rappresenta la quantità di materiale (ad esempio liquido, gas, dati) che viene trasmesso attraverso la rete. Il flusso viene misurato in unità di flusso (ad esempio litri al secondo), e ogni arco nella rete ha una capacità massima di flusso che può trasportare. Il flusso inviato dalla sorgente deve essere uguale al flusso ricevuto dal pozzo, e il flusso attraverso ogni arco non può superare la sua capacità massima.
%
% Il problema di flusso massimo può essere risolto utilizzando l'algoritmo di Ford-Fulkerson, che prevede l'incremento del flusso lungo un cammino aumentante nella rete fino a quando non è possibile trovare ulteriori cammini aumentanti. Un cammino aumentante è un percorso dalla sorgente al pozzo attraverso la rete in cui il flusso attraverso ogni arco non ha ancora raggiunto la capacità massima.
%
% L'algoritmo di Ford-Fulkerson può essere implementato utilizzando l'algoritmo del simplesso sul grafo residuo della rete. Il grafo residuo è un grafo diretto che rappresenta le capacità residue degli archi nella rete, dopo che un certo flusso è stato inviato attraverso la rete. L'algoritmo del simplesso viene utilizzato per trovare il cammino aumentante di capacità massima nel grafo residuo e incrementare il flusso lungo tale cammino.
%
% L'algoritmo di Ford-Fulkerson termina quando non è più possibile trovare cammini aumentanti nel grafo residuo. In quel momento, il flusso trovato è il flusso di massimo valore che può essere trasmesso attraverso la rete.
%
% \subsection{Tagli nella rete}
%
% Un taglio $C$ in una rete di flusso è una partizione del set di nodi $V$ in due sottoinsiemi disgiunti $S$ e $T=V\setminus S$ tali che la sorgente $s$ appartiene a $S$ e il pozzo $t$ appartiene a $T$. Il taglio attraversato da un flusso è il taglio che è attraversato da almeno un arco del flusso. Il valore del taglio $C=(S,T)$ è la somma dei flussi che attraversano tutti gli archi tra $S$ e $T$. In altre parole:
%
% $$
% val(C) = \sum_{i\in S, j\in T} f_{ij}
% $$
%
% dove $f_{ij}$ rappresenta il flusso lungo l'arco $(i,j)$.
%
% Un taglio minimo è un taglio attraverso il quale passa il minimo valore di flusso possibile. In altre parole, è un taglio che ha il valore minimo tra tutti i possibili tagli della rete.
%
% \subsection{Algoritmo di Ford-Fulkerson}
%
% \begin{algorithm}[H]
% \caption{Algoritmo di Ford-Fulkerson}
% \begin{algorithmic}[1]
% \Procedure{Ford-Fulkerson}{$G, s, t$}
% \State Inizializza il flusso $f(u, v) = 0$ per ogni arco $(u,v)\in G$
% \While{esiste un cammino aumentante $p$ in $G_f$}
%     \State Trova la capacità residua minima $c_f(p)$ dell'arco in $p$
%     \ForAll{gli archi $(u,v)\in p$}
%         \State Aggiorna il flusso: $f(u,v) \gets f(u,v) + c_f(p)$
%         \State Aggiorna il flusso inverso: $f(v,u) \gets f(v,u) - c_f(p)$
%     \EndFor
% \EndWhile
% \State \textbf{return} $f$
% \EndProcedure
% \end{algorithmic}
% \end{algorithm}
%
% Nota: $G_f$ indica il grafo residuo corrispondente al flusso $f$. Il cammino aumentante $p$ è un cammino dal nodo sorgente $s$ al nodo destinazione $t$ in $G_f$ tale che la capacità residua $c_f(p)$ sia strettamente positiva per ogni arco in $p$.
%
% \subsection{Procedura di etichettatura}
%
% \begin{enumerate}
%   \item Inizializzare il flusso $f_{u,v} = 0$ per ogni arco $(u,v)$ nel grafo.
%   \item Scegliere un cammino aumentante dal nodo di sorgente $s$ al nodo di destinazione $t$ utilizzando la procedura di ricerca di cammino aumentante come l'algoritmo di Ford-Fulkerson.
%   \item Se non ci sono cammini aumentanti, terminare: il flusso corrente è massimo.
%   \item Calcolare la capacità residua minima lungo il cammino aumentante trovato: 
%   $$c_f(p) = \min_{(u,v) \in p} (c(u,v) - f(u,v))$$
%   dove $p$ è il cammino aumentante.
%   \item Aggiornare il flusso lungo il cammino aumentante: 
%   $$f(u,v) \leftarrow f(u,v) + c_f(p)$$ per ogni arco $(u,v)$ in $p$.
%   \item Aggiornare il flusso lungo gli archi di ritorno del cammino aumentante: 
%   $$f(v,u) \leftarrow f(v,u) - c_f(p)$$ per ogni arco $(v,u)$ in $p$.
%   \item Ritornare al passo 2.
% \end{enumerate}
