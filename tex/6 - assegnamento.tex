\chapter{Assegnamento}

\section{Problema}
L'assegnamento \`e un problema di ottimizzazione che consiste nel trovare la soluzione ottima di un insieme di assegnamenti tra un gruppo di lavoratori e un gruppo di mansioni, considerando i costi associati ad ogni possibile assegnamento. In altre parole, l'assegnamento cerca di minimizzare il costo totale dell'assegnamento dei lavoratori alle mansioni, soddisfacendo allo stesso tempo i vincoli di assegnamento.

\section{Matrice dei costi}
Per questo tipo di task si utilizzano delle matrici che rappresentano i costi,

% ecco un esempio di matrice dei costi per l'assegnamento di lavoratori a mansioni:
\begin{center}
  \begin{tabular}[t]{l|c|c|c|c|c}
        & $b_1$    & $b_2$    & $b_3$    & $b_4$    & $b_5$     \\
  \hline
  $a_1$ & $c_{11}$ & $c_{12}$ & $c_{13}$ & $c_{14}$ & $c_{15}$  \\
  $a_2$ & $c_{21}$ & $c_{22}$ & $c_{23}$ & $c_{24}$ & $c_{25}$  \\
  $a_3$ & $c_{31}$ & $c_{32}$ & $c_{33}$ & $c_{34}$ & $c_{35}$  \\
  $a_4$ & $c_{41}$ & $c_{42}$ & $c_{43}$ & $c_{44}$ & $c_{45}$  \\
  \end{tabular}
\end{center}

\subsection{Riduzione della matrice dei costi}
Per effettuare la riduzione della matrice dei costi occorre selezionare l'elemento
con il costo minore all'interno dell'intera matrice e sottastrarlo a tutti gli altri elementi della matrice.

% esempio con una matrice 4x4
\[
  \begin{array}{cccc}
        & b_1 & b_2 & b_3 \\
    a_1 & 5   & 2   & 4   \\
    a_2 & 2   & 4   & 3   \\
    a_3 & 4   & 3   & 6   \\
  \end{array}
\]

Dove l'elemento con il costo minore \`e $2$ e quindi la matrice diventa:
\[
  \begin{array}{cccc}
      & b_1 & b_2 & b_3 \\
  a_1 & 5   & 2   & 4   \\
  a_2 & 2   & 4   & 3   \\
  a_3 & 4   & 3   & 6   \\
  d_1 & 2   & 2   & 3   \\
  \end{array}
\]


Il risulato di questa operazione \`e una matrice ridotta, che contiene almeno uno zero per ogni riga e colonna.

\[
  \begin{array}{cccc}
      & b_1 & b_2 & b_3 \\
  a_1 & 3   & 0   & 1   \\
  a_2 & 0   & 2   & 0   \\
  a_3 & 2   & 1   & 3   \\
  \end{array}
\]

Questa operazione di riduzione si pu\`o applicare anche aggiungendo una colonna.


\section{Algoritmo ungherese}
L'algoritmo ungherese, noto anche come algoritmo di assegnazione o algoritmo Munkres, è un algoritmo utilizzato per risolvere il problema dell'assegnazione ottima in un problema di assegnazione. Questo problema riguarda l'assegnazione di un insieme di elementi a un altro insieme di elementi, minimizzando o massimizzando una funzione di costo associata.

\textbf{Algoritmo Ungherese (Algoritmo di Assegnazione)}

\begin{algorithm}[H]
\SetAlgoLined
\KwIn{Una matrice dei costi di dimensioni $N \times N$}
\KwOut{Un assegnamento ottimo}

Inizializza una matrice ausiliaria dei costi con gli stessi valori della matrice dei costi\;

\While{Non è stata ottenuta una soluzione ottima}{
  \textbf{Fase 1: Riduzione dei costi}\;
  \For{Ogni riga $i$ della matrice dei costi}{
    Sottrai il valore minimo della riga $i$ a tutti gli elementi della riga\;
  }
  \For{Ogni colonna $j$ della matrice dei costi}{
    Sottrai il valore minimo della colonna $j$ a tutti gli elementi della colonna\;
  }
  
  \textbf{Fase 2: Copertura delle righe e assegnazione}\;
  Copri le righe e assegna le celle seguendo le regole:\;
  \While{Esiste una cella non coperta}{
    Trova una cella non coperta $C$ con il costo minimo nella sua riga e colonna\;
    \If{La riga di $C$ non è coperta}{
      Copri la riga di $C$\;
    }
    Assegna la cella $C$\;
    \While{Esiste una riga coperta che contiene una cella assegnata}{
      Trova una cella assegnata nella stessa riga\;
      Assegna la cella trovata\;
    }
  }
  
  \textbf{Fase 3: Aggiornamento dei costi}\;
  \If{Il numero di assegnazioni effettuate è minore di $N$}{
    Calcola il valore minimo tra gli elementi non coperti della matrice dei costi\;
    Sottrai il valore minimo a tutti gli elementi non coperti\;
    Aggiungi il valore minimo a tutti gli elementi coperti da due linee (riga e colonna)\;
  }
}

\Return{L'assegnamento ottenuto}\;
\caption{Algoritmo Ungherese (Algoritmo di Assegnazione)}
\end{algorithm}

Questo algoritmo risolve il problema dell'assegnazione ottima, riducendo i costi e assegnando le celle in base a determinate regole. Le fasi 1 e 2 vengono eseguite finché non viene ottenuta una soluzione ottima, e la fase 3 viene eseguita solo se necessario. Alla fine, l'algoritmo restituisce un assegnamento ottimo.

