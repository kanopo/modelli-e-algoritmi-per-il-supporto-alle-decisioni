\chapter{Programmazione dinamica}


La programmazione dinamica è una tecnica di risoluzione di problemi che prevede di suddividere il problema in sottoproblemi più piccoli, risolvibili in modo indipendente, e di combinare le soluzioni dei sottoproblemi per ottenere la soluzione del problema originale.

L'approccio si basa sulla costruzione di una tabella in cui le righe rappresentano i sottoproblemi e le colonne rappresentano i valori possibili delle variabili di stato. La tabella viene costruita in modo iterativo, partendo dai casi base e utilizzando i risultati dei sottoproblemi per riempire le celle della tabella.

Il valore della soluzione ottima viene calcolato all'ultimo passaggio della tabella. L'uso della programmazione dinamica è particolarmente utile quando ci sono sovrapposizioni tra i sottoproblemi, cioè quando un sottoproblema viene risolto più volte.

La complessità della programmazione dinamica dipende dal numero di sottoproblemi e dal tempo necessario per risolvere ciascun sottoproblema. In alcuni casi, la tecnica può essere migliorata utilizzando la memoization, ovvero memorizzando i risultati dei sottoproblemi già risolti per evitare di ricalcolarli ogni volta.

In generale, la programmazione dinamica viene utilizzata per risolvere problemi di ottimizzazione in cui la soluzione ottima dipende dalla combinazione di un insieme di decisioni. Tra i problemi risolvibili con la programmazione dinamica si annoverano il problema dello zaino, il problema del commesso viaggiatore e il problema della catena di Markov.

\section{Il principio di ottimalità}
Il principio di ottimalità è un concetto fondamentale della programmazione dinamica. Esso afferma che una soluzione ottima a un problema di ottimizzazione globale può essere costruita attraverso le soluzioni ottime dei suoi sotto-problemi. In altre parole, se un problema può essere suddiviso in sotto-problemi più piccoli, la soluzione ottima del problema globale può essere ottenuta combinando le soluzioni ottime dei suoi sotto-problemi.

Questo principio si applica quando si hanno problemi in cui la soluzione ottima di una istanza del problema contiene al suo interno la soluzione ottima di sotto-istanze del problema stesso. In questo caso, la soluzione del problema può essere ottenuta risolvendo le sotto-istanze e combinando le loro soluzioni in modo opportuno.

L'utilizzo del principio di ottimalità permette di evitare di risolvere più volte lo stesso sotto-problema, riducendo così il tempo di calcolo e aumentando l'efficienza dell'algoritmo di programmazione dinamica.


\section{Programmazione dinamica per il problema dello zaino}


Si consideri il problema dello zaino in cui si ha a disposizione uno zaino di capacità $C$ e un insieme di $n$ oggetti. Ogni oggetto $i$ ha un peso $p_i$ e un valore $v_i$. Si vuole trovare la combinazione di oggetti che massimizza il valore totale, rispettando la capacità dello zaino.

Definiamo $K(i, w)$ la soluzione ottima del problema dello zaino utilizzando i primi $i$ oggetti e uno zaino di capacità $w$. Il problema può essere risolto tramite programmazione dinamica utilizzando il principio di ottimalità.

Il principio di ottimalità afferma che una soluzione ottima al problema dello zaino che considera i primi $i$ oggetti, è ottenuta considerando o l'oggetto $i$ o meno. Quindi, la soluzione ottima può essere ottenuta confrontando il valore massimo che si può ottenere considerando l'oggetto $i$ (e utilizzando uno zaino di capacità $w-p_i$) con il valore massimo che si può ottenere senza considerare l'oggetto $i$ (e utilizzando uno zaino di capacità $w$). Formalmente:

\[
K(i, w) = 
\begin{cases} 
0 & i = 0 \text{ o } w = 0 \\
K(i-1, w) & p_i > w \\
\max \{ K(i-1, w), K(i-1, w-p_i) + v_i \} & \text{altrimenti}
\end{cases}
\]

La soluzione al problema dello zaino può essere trovata calcolando $K(n, C)$.

Inoltre, è possibile utilizzare la programmazione dinamica per trovare la combinazione di oggetti che massimizza il valore totale. Dopo aver calcolato la matrice $K(i,w)$, si può risalire ai singoli oggetti utilizzati nella soluzione ottima tramite la seguente procedura:

\begin{algorithm}[H]
\SetAlgoLined
\KwResult{Oggetti utilizzati nella soluzione ottima}
$i \leftarrow n$\;
$w \leftarrow C$\;
\While{$i > 0$}{
  \If{$K(i,w) \neq K(i-1,w)$}{
    Utilizza l'oggetto $i$\;
    $w \leftarrow w - p_i$\;
  }
  $i \leftarrow i - 1$\;
}
\caption{Procedura per ottenere gli oggetti utilizzati nella soluzione ottima}
\end{algorithm}


\section{Che cos'è l'algoritmo valore ottimo?}

L'algoritmo valore ottimo (in inglese \textit{Value Iteration}) è un algoritmo di programmazione dinamica utilizzato per trovare la politica ottima in un processo decisionale di Markov a tempo discreto (MDP). L'algoritmo è basato sull'iterazione dei valori e utilizza una procedura di backup dei valori per aggiornare il valore di ogni stato.

L'idea alla base dell'algoritmo valore ottimo è quella di trovare la funzione valore ottimo $V^*(s)$ di ogni stato $s$ del MDP. Questa funzione indica il valore atteso della ricompensa totale che si può ottenere partendo dallo stato $s$ e seguendo la politica ottima. L'algoritmo itera la stima della funzione valore ottimo fino a raggiungere una convergenza. Alla fine di ogni iterazione, l'algoritmo aggiorna la funzione valore ottimo di ogni stato sulla base del valore atteso dei suoi successori.

L'algoritmo valore ottimo è un metodo molto efficace per risolvere MDPs di grandi dimensioni, tuttavia richiede la conoscenza completa del modello del sistema, ovvero delle probabilità di transizione e delle ricompense associate ad ogni transizione.


\subsection{Applicato a KNAPSACK}

\begin{algorithm}[H]
\SetAlgoLined
\KwIn{Un insieme di $n$ oggetti, dove ogni oggetto $i$ ha un valore $v_i$ e un peso $w_i$, e una capacità massima dello zaino $W$}
\KwOut{Il valore massimo che può essere ottenuto con una combinazione di oggetti che non superi la capacità $W$}

$A \gets$ array $n \times (W+1)$ inizializzato a 0\;

\For{$i \gets 1$ \KwTo $n$}{
    \For{$w \gets 1$ \KwTo $W$}{
        \eIf{$w_i \leq w$}{
            $A[i, w] \gets \max\{A[i-1, w], A[i-1, w-w_i]+v_i\}$\;
        }{
            $A[i, w] \gets A[i-1, w]$\;
        }
    }
}
\Return{$A[n, W]$}
\caption{Algoritmo valore ottimo per il problema dello zaino}
\label{alg:valore-ottimo-zaino}
\end{algorithm}

In questo algoritmo, si utilizza una matrice $A$ di dimensioni $n \times (W+1)$ per memorizzare i valori ottimi per tutti i sottoproblemi del problema dello zaino, ovvero il valore massimo che può essere ottenuto con una combinazione di oggetti che non superi una capacità $w \leq W$ e che includa solo i primi $i$ oggetti.

L'algoritmo utilizza una doppia iterazione sui primi $i$ oggetti e sulle capacità $w$. Se l'oggetto $i$ ha peso $w_i$ minore o uguale alla capacità $w$ considerata, allora è possibile scegliere se includere o meno l'oggetto $i$ nella combinazione. Se si decide di includerlo, il valore massimo ottenibile è dato dalla somma del valore dell'oggetto $i$ e del valore massimo ottenibile con i primi $i-1$ oggetti e una capacità residua di $w-w_i$. Altrimenti, il valore massimo ottenibile è dato dal valore massimo ottenibile con i primi $i-1$ oggetti e la capacità $w$ considerata.

Alla fine delle iterazioni, il valore ottimo per il problema originale, ovvero la combinazione di oggetti con valore massimo che non superi la capacità $W$, è dato dall'elemento $A[n, W]$ della matrice.

\section{Problema della schedulazione}

Il \textbf{problema della schedulazione} consiste nell'assegnare un insieme di \textit{n} attività da eseguire, ognuna delle quali richiede un certo tempo di elaborazione, a un insieme di \textit{m} risorse, ciascuna delle quali ha una certa disponibilità. L'obiettivo è minimizzare il tempo totale di completamento delle attività.

Formalmente, il problema può essere definito come segue:

Dati:
\begin{itemize}
    \item Un insieme di \textit{n} attività $A = \{a_1, a_2, ..., a_n\}$.
    \item Un insieme di \textit{m} risorse $R = \{r_1, r_2, ..., r_m\}$.
    \item Un tempo di elaborazione $p_{ij}$ per eseguire l'attività $a_i$ sulla risorsa $r_j$.
    \item Una disponibilità $d_j$ per ogni risorsa $r_j$.
\end{itemize}
Obiettivo: Trovare un assegnamento delle attività alle risorse che minimizzi il tempo totale di completamento.

Il problema della schedulazione può essere formulato in diverse varianti, come ad esempio il problema della schedulazione su una singola macchina o il problema della schedulazione su più macchine. In generale, il problema è considerato \textit{NP-hard}, il che significa che non esiste un algoritmo efficiente in grado di risolverlo in tempo polinomiale per tutti i casi.
