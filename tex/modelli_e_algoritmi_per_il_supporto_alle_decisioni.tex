\documentclass{report}

% ----------------------------------------------------
% PACKAGE
% ----------------------------------------------------

\usepackage{graphicx}
\usepackage{amsmath}
\usepackage{amsfonts}
\usepackage{amssymb}
\usepackage{hyperref}
\usepackage[a4paper, portrait, margin=0.75in]{geometry}
\usepackage[italian]{babel}
\usepackage{enumerate}% http://ctan.org/pkg/enumerate

\usepackage{tcolorbox} % usato per fare le box colorate
\usepackage{soul} % usato per barrare il testo
% \usepackage{algorithm} % usato per scrivere algoritmi
\usepackage{algpseudocode}
\usepackage{algorithm2e}
\usepackage{tikz}
\usepackage{listings}

\lstset{
  language=Python,
  basicstyle=\ttfamily,
  keywordstyle=\color{blue},
  stringstyle=\color{red},
  commentstyle=\color{green},
  numbers=left,
  numberstyle=\tiny\color{gray},
  breaklines=true,
  frame=single,
  captionpos=b,
  showstringspaces=false,
  tabsize=2
}
\hypersetup{
    colorlinks=true,
    linkcolor=black,
    urlcolor=blue,
}

\title{Modelli e algoritmi per il supporto delle decisioni \\[1ex] \large Un goliardico riassunto}
\author{Ollari Dmitri}

\begin{document}
    \maketitle

    \tableofcontents
    % \listoffigures

    \chapter{Introduzione ai modelli}
\section{Tipologie di algoritmi}

\begin{enumerate}
  \item Algoritmi costruttivi: Gli algoritmi costruttivi vengono utilizzati per costruire una soluzione in modo incrementale, aggiungendo passo dopo passo componenti o elementi al problema. Questi algoritmi partono da una soluzione vuota e aggiungono iterativamente elementi considerando regole o euristiche specifiche. Gli algoritmi costruttivi sono spesso efficienti dal punto di vista computazionale, ma potrebbero non garantire la soluzione ottimale. Tuttavia, possono essere utili quando è richiesta una soluzione veloce o approssimata.
  \item Algoritmi di enumerazione: Gli algoritmi di enumerazione esplorano in modo sistematico tutte le possibili soluzioni di un problema, esaminando ogni possibile combinazione. Questi algoritmi possono essere implementati utilizzando tecniche come l'albero delle decisioni o la generazione di tutte le permutazioni. Gli algoritmi di enumerazione possono garantire la completezza (esaminando tutte le soluzioni) ma possono richiedere un tempo di esecuzione elevato per problemi di dimensioni significative.
  \item Algoritmi di raffinamento locale: Gli algoritmi di raffinamento locale si concentrano sulla ricerca di soluzioni ottimali all'interno di un'area di ricerca limitata del problema. Questi algoritmi partono da una soluzione iniziale e iterativamente esplorano soluzioni vicine, apportando miglioramenti incrementali fino a quando non viene raggiunta una soluzione che soddisfa i criteri di ottimalità specificati. Gli algoritmi di raffinamento locale sono particolarmente efficaci per problemi in cui una piccola modifica nella soluzione può portare a un miglioramento sostanziale.
\end{enumerate}


    \chapter{Introduzione ai grafi}

\textbf{Definizione:} Un grafo è una struttura di dati utilizzata per rappresentare le relazioni tra oggetti. Formalmente, un grafo G è una coppia ordinata G = (V, E), dove V rappresenta l'insieme dei nodi (o vertici) e E rappresenta l'insieme degli archi. Gli archi possono essere diretti o non diretti, a seconda che rappresentino una connessione unidirezionale o bidirezionale tra i nodi.

\textbf{Terminologia:}
\begin{itemize}
  \item \textbf{Grado di un nodo:} Il grado di un nodo è il numero di archi che sono connessi ad esso.
  \item \textbf{Percorso:} Un percorso è una sequenza di nodi collegati da archi.
  \item \textbf{Ciclo:} Un ciclo è un percorso che inizia e termina nello stesso nodo.
  \item \textbf{Grafo connesso:} Un grafo connesso è un grafo in cui esiste un percorso tra ogni coppia di nodi.
  \item \textbf{Grafo pesato:} Un grafo pesato è un grafo in cui ogni arco ha un peso o un valore associato.
  \item \textbf{Grafo orientato:} Un grafo orientato è un grafo in cui gli archi hanno una direzione specifica.
\end{itemize}

\textbf{Rappresentazioni:}
\begin{itemize}
  \item \textbf{Matrice di adiacenza:} Una matrice bidimensionale che rappresenta la presenza o l'assenza di un arco tra i nodi.
  \item \textbf{Lista di adiacenza:} Una lista in cui ogni nodo ha un elenco dei suoi nodi adiacenti.
  \item \textbf{Lista degli archi:} Una lista che contiene tutte le coppie di nodi connessi da un arco.
\end{itemize}

La teoria dei grafi offre un ampio spettro di algoritmi e tecniche per lo studio e l'analisi dei grafi. Tra questi, ci sono algoritmi di attraversamento del grafo come la ricerca in profondità (DFS) e la ricerca in ampiezza (BFS), algoritmi per il percorso più breve come l'algoritmo di Dijkstra e algoritmi per la costruzione di alberi di copertura minimi come l'algoritmo di Kruskal.

La teoria dei grafi è una disciplina fondamentale con numerose applicazioni pratiche, che permette di risolvere una vasta gamma di problemi e modellare relazioni complesse tra gli oggetti.
\section{Rappresentazione grafica}
\begin{center}
  \begin{tikzpicture}
    % Nodi
    \node[draw, circle] (A) at (0,0) {A};
    \node[draw, circle] (B) at (2,2) {B};
    \node[draw, circle] (C) at (4,0) {C};
    \node[draw, circle] (D) at (2,-2) {D};
    
    % Archi
    \draw[->] (A) -- (B);
    \draw[->] (B) -- (C);
    \draw[->] (C) -- (D);
    \draw[->] (C) -- (A);
    \draw[->] (D) -- (A);
  \end{tikzpicture}
\end{center}

\section{Liste di adiacenza}

Le liste di adiacenza sono una delle rappresentazioni più comuni per i grafi. In questa rappresentazione, ogni nodo del grafo è associato a una lista di nodi adiacenti, cioè i nodi che sono direttamente collegati a quel nodo.

Per capire meglio, consideriamo un esempio di un grafo non diretto con 4 nodi (A, B, C, D) e 5 archi (AB, AC, AD, BC, CD). Ecco come potrebbe essere rappresentato utilizzando le liste di adiacenza:

\begin{center}
  \begin{verbatim}
  A: [B, C, D]
  B: [A, C]
  C: [A, B, D]
  D: [A, C]
  \end{verbatim}
\end{center}
Nell'esempio sopra, ogni nodo è seguito da una lista di nodi adiacenti. Ad esempio, il nodo A ha tre nodi adiacenti, che sono B, C e D.

La rappresentazione tramite liste di adiacenza presenta alcuni vantaggi. In primo luogo, è efficiente per i grafi sparsi, cioè quei grafi in cui il numero di archi è molto inferiore al numero massimo possibile di archi. Inoltre, consente di accedere rapidamente ai nodi adiacenti di un dato nodo. Tuttavia, può richiedere più spazio di memoria rispetto ad altre rappresentazioni come le matrici di adiacenza se il grafo ha molti archi.

Le liste di adiacenza sono utilizzate in numerosi algoritmi di grafi, come la ricerca in ampiezza (BFS) e la ricerca in profondità (DFS), poiché consentono un accesso efficiente ai vicini di un nodo.

\section{Matrici di incidenza nodo arco}
Le matrici di incidenza nodo-arco sono una rappresentazione comune per i grafi. In questa rappresentazione, le righe della matrice rappresentano i nodi del grafo e le colonne rappresentano gli archi. Gli elementi della matrice indicano l'incidenza dei nodi sugli archi, cioè se un nodo è collegato o meno a un particolare arco.

Per comprendere meglio, consideriamo un esempio di un grafo diretto con 4 nodi (A, B, C, D) e 5 archi (AB, AC, AD, BC, CD). Ecco come potrebbe essere rappresentato utilizzando una matrice di incidenza nodo-arco:

\begin{center}
\begin{tabular}{c|ccccc}
    & AB & AC & AD & BC & CD \\
\hline
A & -1 & 0 & 0 & 0 & 0 \\
B & 1 & -1 & 0 & 0 & 0 \\
C & 0 & 1 & -1 & 0 & 0 \\
D & 0 & 0 & 1 & -1 & 1 \\
\end{tabular}
\end{center}


Nell'esempio sopra, ogni riga della matrice rappresenta un nodo, mentre ogni colonna rappresenta un arco. Gli elementi della matrice possono assumere i seguenti valori:
\begin{itemize}
  
  \item -1: indica che il nodo è la sorgente dell'arco.
  \item 1: indica che il nodo è la destinazione dell'arco.
  \item 0: indica che il nodo non è collegato all'arco.
\end{itemize}

La rappresentazione tramite matrici di incidenza nodo-arco è particolarmente utile per i grafi con archi diretti e pesati. Questa rappresentazione consente di visualizzare facilmente le connessioni tra i nodi e gli archi del grafo. Tuttavia, rispetto alle liste di adiacenza o alle matrici di adiacenza, le matrici di incidenza richiedono più spazio di memoria e possono essere meno efficienti per alcune operazioni, come la ricerca dei vicini di un nodo.

Le matrici di incidenza nodo-arco sono utilizzate in diversi algoritmi e applicazioni dei grafi, come la risoluzione dei problemi di flusso massimo/minimo e la modellazione di reti di trasporto e comunicazione.


\section{Archi adiacenti e cammini}
Gli archi adiacenti si riferiscono agli archi che condividono un nodo in comune. In altre parole, due archi sono adiacenti se hanno un'estremità comune, che può essere un nodo di partenza o un nodo di arrivo. Ad esempio, se abbiamo un grafo con archi AB, AC e BC, gli archi AB e AC sono adiacenti poiché entrambi hanno il nodo A come estremità iniziale.

I cammini, d'altra parte, sono sequenze ordinate di nodi collegati da archi. Un cammino in un grafo è una serie di nodi in cui ogni coppia consecutiva di nodi è collegata da un arco. Possiamo distinguere diversi tipi di cammini:

\begin{itemize}
  \item Cammino semplice: un cammino in cui nessun nodo appare più di una volta.
  \item Cammino elementare: un cammino in cui nessun arco appare più di una volta.
  \item Cammino ciclo: un cammino in cui il nodo di partenza coincide con il nodo di arrivo.
\end{itemize}

Ad esempio, considera un grafo con i nodi A, B, C e D e gli archi AB, BC e CD. Un possibile cammino in questo grafo potrebbe essere AB-BC-CD, che rappresenta un cammino semplice che connette il nodo A al nodo D attraverso i nodi B e C.

\section{Circuiti hamiltoniani}
Un circuito hamiltoniano in un grafo è un cammino chiuso che attraversa ogni nodo esattamente una volta, ad eccezione del nodo di partenza/arrivo che viene visitato due volte. In altre parole, è un percorso che visita tutti i nodi del grafo una sola volta, ritornando infine al nodo di partenza.

\section{Componenti connesse}
Le componenti connesse sono gruppi di nodi in un grafo che sono collegati tra loro attraverso archi. In altre parole, una componente connessa è un sottoinsieme di nodi di un grafo in cui esiste un percorso tra ogni coppia di nodi all'interno di quella componente.


\subsection{Trovare le componenti connesse}
\begin{lstlisting}
def find_components(graph):
    visited = set()
    components = []

    def dfs(node, component):
        visited.add(node)
        component.append(node)

        for neighbor in graph[node]:
            if neighbor not in visited:
                dfs(neighbor, component)

    for node in graph:
        if node not in visited:
            current_component = []
            dfs(node, current_component)
            components.append(current_component)

    return components

graph = {
    'A': ['B', 'C'],
    'B': ['A', 'C'],
    'C': ['A', 'B'],
    'D': ['E'],
    'E': ['D']
}

components = find_components(graph)
print(components)
\end{lstlisting}


\section{Grafo completo}
Un grafo completo è un tipo particolare di grafo non diretto in cui ogni coppia di nodi è collegata da un arco. In altre parole, in un grafo completo, ogni nodo è connesso direttamente a tutti gli altri nodi del grafo.


\section{Matching}

In teoria dei grafi, un \textit{matching} in un grafo $G = (V, E)$ è un insieme di archi $M \subseteq E$ tale che nessun nodo condivide gli estremi di due archi in $M$.

Formalmente, dato un grafo non diretto $G = (V, E)$, un matching $M$ in $G$ soddisfa le seguenti condizioni:
\begin{enumerate}
    \item Per ogni arco $(u, v) \in M$, nessun arco in $M$ condivide il nodo iniziale $u$ o il nodo finale $v$.
    \item Ogni nodo in $V$ è incidente a al più un arco in $M$.
\end{enumerate}

Un \textit{matching massimale} è un matching che non può essere esteso aggiungendo ulteriori archi senza violare la proprietà di non sovrapposizione. Un \textit{matching perfetto} è un matching in cui tutti i nodi del grafo sono coperti da un arco del matching.

Il matching ha diverse applicazioni, come problemi di assegnazione, progettazione di reti, teoria dei giochi e altro ancora.

\section{Grafi bipartiti}

Un grafo bipartito è un tipo di grafo non diretto in cui i nodi possono essere divisi in due insiemi disgiunti, in modo che ogni arco del grafo colleghi un nodo di un insieme all'altro.

Formalmente, un grafo bipartito $G = (V, E)$ è definito da due insiemi di nodi disgiunti $V1$ e $V2$, dove gli archi del grafo collegano solo i nodi di $V1$ con i nodi di $V2$.

I grafi bipartiti sono rappresentati graficamente come insiemi di punti su due righe parallele, dove gli archi collegano i punti delle due righe.

I grafi bipartiti hanno diverse proprietà interessanti, come la possibilità di essere colorati con solo due colori e l'assenza di cicli di lunghezza dispari.

\subsection{Trovare grafi bipartiti}
\begin{lstlisting}
 
def is_bipartite(graph):
    color = {}
    visited = set()

    def dfs(node, c):
        visited.add(node)
        color[node] = c

        for neighbor in graph[node]:
            if neighbor not in visited:
                if not dfs(neighbor, 1 - c):
                    return False
            elif color[neighbor] == color[node]:
                return False

        return True

    for node in graph:
        if node not in visited:
            if not dfs(node, 0):
                return False

    return True


graph = {
    'A': ['B', 'C'],
    'B': ['A', 'D'],
    'C': ['A', 'D'],
    'D': ['B', 'C']
}

if is_bipartite(graph):
  print("bipartito")
else:
  print("non bipartito")
\end{lstlisting}


\section{Alberi}


Un albero è una struttura dati gerarchica che consiste in un insieme di nodi collegati tra loro in modo specifico. L'albero è composto da una radice, nodi, e archi che connettono i nodi.

I principali termini associati agli alberi sono:

\begin{itemize}
    \item \textbf{Radice}: il nodo di partenza dell'albero.
    \item \textbf{Nodi}: gli elementi dell'albero.
    \item \textbf{Archi}: i collegamenti tra i nodi dell'albero.
    \item \textbf{Figli}: i nodi direttamente collegati a un dato nodo.
    \item \textbf{Genitore}: il nodo da cui un dato nodo è raggiungibile tramite un arco.
    \item \textbf{Foglie}: i nodi che non hanno figli.
    \item \textbf{Livello}: la distanza tra un nodo e la radice.
    \item \textbf{Altezza}: il numero massimo di livelli dell'albero.
\end{itemize}

Gli alberi sono utilizzati in diverse applicazioni, come la rappresentazione delle directory nei sistemi operativi, la gerarchia delle organizzazioni aziendali e la struttura dei dati nelle basi di dati.

\section{Alberi di supporto}
Gli alberi di supporto, noti anche come alberi di copertura o alberi di connessione minima, sono alberi speciali utilizzati in teoria dei grafi. Sono utilizzati per connettere tutti i nodi di un grafo con un numero minimo di archi, garantendo al contempo la connessione tra tutti i nodi.

Formalmente, dato un grafo non diretto ponderato $G = (V, E)$, dove $V$ rappresenta l'insieme dei nodi e $E$ rappresenta l'insieme degli archi, un albero di supporto di $G$ è un sottografo che soddisfa le seguenti proprietà:

\begin{enumerate}
    \item L'albero di supporto contiene tutti i nodi di $G$.
    \item L'albero di supporto è un albero, cioè non contiene cicli.
    \item L'albero di supporto ha il minor peso totale tra tutti gli alberi che soddisfano le prime due proprietà.
\end{enumerate}

Gli alberi di supporto sono utili in diversi contesti, come reti di comunicazione, logistica, progettazione di reti stradali e molto altro ancora.

    \chapter{Minimum Spanning Tree}

Un Minimum Spanning Tree (MST), tradotto in italiano come Albero di Supporto Minimo, è un albero di supporto di peso minimo in un grafo non diretto e connesso. L'MST è costituito da un sottoinsieme di archi del grafo che connette tutti i nodi in modo tale che il peso totale degli archi sia il più basso possibile.

Le caratteristiche principali degli MST sono:

\begin{enumerate}
    \item \textbf{Connessione}: Un MST deve connettere tutti i nodi del grafo, garantendo che non ci siano nodi isolati.
    \item \textbf{Aciclicità}: Un MST non deve contenere cicli, quindi non può avere archi che creano loop all'interno dell'albero.
    \item \textbf{Peso minimo}: Un MST ha il peso totale degli archi più basso possibile tra tutti gli alberi che soddisfano le prime due caratteristiche.
\end{enumerate}

Gli MST hanno diverse applicazioni pratiche, tra cui:

\begin{itemize}
    \item Reti di comunicazione: Un MST può essere utilizzato per collegare un insieme di punti in una rete di comunicazione minimizzando il costo totale dei collegamenti.
    \item Progettazione di reti stradali: Gli MST possono essere utilizzati per pianificare reti stradali efficienti, dove gli archi rappresentano le strade e il peso degli archi può essere la distanza o il tempo di percorrenza.
    \item Analisi dei dati: Gli MST possono essere utilizzati per individuare le relazioni più rilevanti o significative tra i dati, ad esempio nella visualizzazione delle relazioni tra punti di dati su una mappa.
\end{itemize}


\section{Algoritmo greedy per MST}
L'algoritmo greedy per MST (Minimum Spanning Tree) è un approccio basato sulla selezione di archi in base al loro peso. L'idea principale è quella di selezionare ripetutamente l'arco di peso minimo che collega un nodo dell'MST esistente a un nodo non ancora raggiunto, finché non viene creato un albero che connette tutti i nodi del grafo.



\begin{enumerate}
  
  \item Inizializzazione: Parti da un grafo non diretto e connesso $G = (V, E)$, dove $V$ rappresenta l'insieme dei nodi e $E$ rappresenta l'insieme degli archi. Crea un insieme vuoto MST che conterrà gli archi dell'albero di supporto minimo.
  \item Seleziona un nodo di partenza arbitrario. Questo può essere fatto in modo casuale o seguendo una strategia specifica, ad esempio selezionando il primo nodo dell'insieme dei nodi.
  \item Finché non sono stati raggiunti tutti i nodi:
    \begin{enumerate}
      \item Seleziona l'arco di peso minimo che collega un nodo nell'MST esistente a un nodo non ancora raggiunto. Questo arco deve essere selezionato tra gli archi che collegano il nodo raggiunto all'esterno dell'MST.
      \item Aggiungi l'arco selezionato all'MST.
      \item Marca il nodo raggiunto come "visitato" o "raggiunto".
    \end{enumerate}
  \item Alla fine del processo, l'MST conterrà tutti gli archi necessari per connettere tutti i nodi del grafo in modo che il peso totale dell'MST sia minimo.
\end{enumerate}
L'algoritmo greedy per MST può essere implementato utilizzando diverse strutture dati, come ad esempio una coda di priorità (heap) per selezionare l'arco di peso minimo in modo efficiente.

Ecco un esempio di implementazione dell'algoritmo greedy per MST in Python:

\begin{lstlisting}
  
def greedy_mst(graph):
    mst = []  # Insieme di archi dell'MST
    visited = set()  # Insieme di nodi visitati
    start_node = list(graph.keys())[0]  # Nodo di partenza arbitrario
    
    visited.add(start_node)
    
    while len(visited) < len(graph):
        min_weight = float('inf')
        min_edge = None
        
        # Scansiona gli archi collegati ai nodi visitati
        for node in visited:
            for neighbor, weight in graph[node]:
                if neighbor not in visited and weight < min_weight:
                    min_weight = weight
                    min_edge = (node, neighbor)
        
        if min_edge:
            mst.append(min_edge)
            visited.add(min_edge[1])
    
    return mst
\end{lstlisting}

Questo è solo un esempio di implementazione e può variare a seconda delle specifiche del problema. Assicurati di adattare l'algoritmo in base alle tue esigenze specifiche.

\subsection{Correttezza algoritmo}


La complessità dell'algoritmo greedy per MST dipende dalla rappresentazione del grafo e dalla struttura dati utilizzata.

Assumendo che il grafo sia rappresentato come una lista di adiacenza, con $n$ nodi e $m$ archi, e che si utilizzi una coda di priorità (heap) per selezionare l'arco di peso minimo, la complessità dell'algoritmo è:

\begin{itemize}
    \item Inizializzazione: $O(n)$
    \item Ciclo principale: $O(m \log m)$
\end{itemize}

All'interno del ciclo principale, l'estrazione dell'arco di peso minimo richiede un'operazione di estrazione minima dalla coda di priorità, che ha una complessità di $O(\log m)$. Il ciclo principale viene eseguito $m$ volte.

Quindi, la complessità complessiva dell'algoritmo greedy per MST è $O(n + m \log m)$.

È importante notare che se il grafo viene rappresentato come una matrice di adiacenza invece di una lista di adiacenza, la complessità può essere diversa, ad esempio $O(n^2)$ per la ricerca dell'arco di peso minimo.
\subsection{Complessit\'a dell'algoritmo}






\section{Algoritmo MST-1 - Prim}

\begin{center}
\begin{tikzpicture}[auto,node distance=2.5cm]
    \node (A) {A};
    \node (B) [below left of=A] {B};
    \node (C) [below right of=A] {C};
    \node (D) [below right of=B] {D};
    \node (E) [right of=C] {E};
    \node (F) [below right of=E] {F};
    \node (G) [below left of=F] {G};

    \path (A) edge node {2} (B)
              edge node {5} (C)
          (B) edge node {1} (D)
              edge node {3} (C)
          (C) edge node {2} (D)
              edge node {4} (E)
          (D) edge node {3} (F)
          (E) edge node {1} (F)
          (F) edge node {2} (G);
\end{tikzpicture}\end{center}


  \subsection{Procedimento}
  \begin{enumerate}
    \item Sceglie un nodo qualsiasi come nodo di partenza e lo aggiunge alla lista dei nodi visitati
    \item Scegliere un nodo non visitato che abbia un arco che lo colleghi a un nodo visitato con peso minimo e aggiungerlo alla lista dei nodi visitati
    \item Ripetere il passo 2 fino a quando non sono stati visitati tutti i nodi
  \end{enumerate}
  



\section{Algoritmo MST-2 - Kruskal}

\begin{center}
\begin{tikzpicture}[auto,node distance=2.5cm]
    \node (A) {A};
    \node (B) [below left of=A] {B};
    \node (C) [below right of=A] {C};
    \node (D) [below right of=B] {D};
    \node (E) [right of=C] {E};
    \node (F) [below right of=E] {F};
    \node (G) [below left of=F] {G};

    \path (A) edge node {2} (B)
              edge node {5} (C)
          (B) edge node {1} (D)
              edge node {3} (C)
          (C) edge node {2} (D)
              edge node {4} (E)
          (D) edge node {3} (F)
          (E) edge node {1} (F)
          (F) edge node {2} (G);
\end{tikzpicture}\end{center}


  \subsection{Procedimento}
  \begin{enumerate}
    \item Si ordinano tutti gli archi del grafo in ordine crescente del peso.
    \item Si aggiunge un arco alla volta alla lista degli archi dell'albero di supporto, a patto che non si creino cicli.
    \item Si ripete il passo 2 fino a che non si sono aggiunti $n-1$ archi, dove $n$ è il numero di nodi del grafo.
  \end{enumerate}








  \chapter{Algoritmi di cammini minimi}


  Ci sono due algotimi principali per risolvere il problema del cammino minimo:
  \begin{itemize}
    \item Algoritmo di Dijkstra: valido per grafi con pesi non negativi
    \item Algoritmo di Floyd-Warshall: valido per grafi con anche pesi negativi
  \end{itemize}


  \section{Algoritmo di Dijkstra}
  Percorso minimo da un nodo $s$ a tutti i nodi del grafo(grafo con pesi non negativi).

  \begin{enumerate}
    \item Si sceglie un nodo di partenza $s$ e si inizializza un vettore $d$ con la distanza di tutti i nodi da $s$.
    \item Si sceglie il nodo $v$ con distanza minima da $s$ e si aggiorna il vettore $d$ con le distanze dei nodi adiacenti a $v$.
    \item Si ripete il passo 2 fino a che non sono stati visitati tutti i nodi.
  \end{enumerate}


  \subsection{Complessit\`a}
  L'algoritmo di Dijkstra ha complessit\`a $O(n^2)$ se il grafo \`e rappresentato con una matrice di adiacenza, $O(m \log n)$ se il grafo \`e rappresentato con una lista di adiacenza.



  \section{Algoritmo di Floyd-Warshall}
  Percorso minimo tra tutti i nodi del grafo(grafo con pesi anche negativi).

  DA RIFARE

%   \begin{enumerate}
%     \item Inizializziamo una matrice di dimensioni N x N, dove N è il numero di nodi nel grafo. Chiamiamo questa matrice "dist" e inizializziamo i suoi valori con i pesi degli archi del grafo. Se non esiste un arco tra due nodi, il valore corrispondente nella matrice sarà infinito.
%     \item Per ogni nodo nel grafo, impostiamo la distanza da se stesso a 0 nella matrice "dist".
%     \item Eseguiamo un triplo ciclo for per tutti i nodi del grafo. Il primo ciclo indica il nodo intermedio che stiamo considerando per il calcolo dei cammini minimi.
%     \item All'interno del triplo ciclo for, confrontiamo la distanza tra il nodo di partenza e il nodo di destinazione attraverso il nodo intermedio con la distanza attuale tra il nodo di partenza e il nodo di destinazione. Se la distanza attraverso il nodo intermedio è minore, aggiorniamo il valore corrispondente nella matrice "dist".
%     \item Alla fine dell'algoritmo, la matrice "dist" conterrà le distanze minime tra tutti i nodi del grafo.
%     \item Possiamo utilizzare la matrice "dist" per determinare i cammini minimi tra tutti i nodi del grafo. Inoltre, possiamo individuare cicli negativi nel grafo verificando se un valore diagonale nella matrice "dist" è negativo.
%   \end{enumerate}
%
% L'algoritmo di Floyd-Warshall è efficiente per grafi di piccole dimensioni, poiché richiede un tempo di esecuzione di $O(N^3)$, dove $N$ è il numero di nodi nel grafo. Tuttavia, può diventare inefficiente per grafi di grandi dimensioni a causa della complessità cubica.
%
% \section{Shortest path}
%
% Nei problemi di cammino a costo minimo, dadto un grafo $G=(V,A)$ con costo (distanza) $d_{ij}$ per ogni 
% $(i, j) \in A$ e dati due nodi $s, t \in V$ dove i due nodi sono diversi fra di loro, vogliamo individuare un cammino
% elementare orientato da $s$ a $t$ di costo minomo.
%
% \subsection{Algoritmo di Dijkstra}
% Valido solo se $d_{ij} \geq 0 \forall (i, j) \in A$.
% Restituisce i cammini minimi tra un nodo scelto $s \in V$ e tutti gli altri nodi.
%
% \subsection{Algoritmo di Floyd-Warshall}
% Valido anche per distanze negative a patto che non siano presenti cicli a costo negativo(in questo caso restituisce un ciclo a costo negativo).
%
%

    % TODO: capire se l'algoritmo ungherese ha senso nei matching
    % per il momento lo lascio solo nell'assegnazione
    % 
\chapter{Matching}

    \chapter{Problemi di flusso}

\section{Algoritmi per i problemi di flusso a costo minimo}

Gli algoritmi per i problemi di flusso a costo minimo sono un insieme di tecniche utilizzate per trovare il flusso a costo minimo in un grafo. Il flusso a costo minimo si riferisce alla quantità massima di materiale che può essere spostata da una sorgente a una destinazione attraverso un grafo con il costo minimo.


\subsection{Soluzione di base}
la soluzione di base si ottiene ponenendo il flusso di tutti gli archi che fanno parte
dell'albero di supporto a 0.

\subsection{Ammissibilità e degenerazione}
Se i flussi in base hanno valore \textbf{non negativo}, si parla di base \textbf{ammissibile.}
Se i flussi in base hanno valore \textbf{nullo}, si parla di base \textbf{degenere.}

\subsection{Coefficiente di costo ridotto}
Data una soluzione di base ammissibile, viene associato un \textbf{coefficiente di costo ridotto} che misura
la variazioe del valore dell'obiettivo al crescere dell'unità del valore del flusso su un arco 
fuori base.

\subsection{Condizione di ottimalità}
Assumiamo di avere una soluzione di base per un problema di programmazione lineare a costo minimo con vincoli di flusso. Una soluzione di base è costituita da un insieme di archi che formano un flusso ammissibile, mentre tutti gli altri archi sono fuori base, cioè non fanno parte del flusso.

Il coefficiente di costo ridotto di un arco è la differenza tra il costo dell'arco e il costo marginale del flusso su quell'arco. In altre parole, il costo ridotto di un arco misura di quanto si ridurrebbe il costo totale del flusso se si aumentasse il flusso su quell'arco di un'unità.

Se i coefficienti di costo ridotto di tutti gli archi fuori base sono non negativi, significa che aumentare il flusso su qualsiasi arco fuori base dal suo valore nullo attuale comporta una crescita dell'obiettivo (se il coefficiente è positivo) o nessuna variazione nell'obiettivo (se ha valore nullo).

Questo accade perché se il coefficiente di costo ridotto di un arco fuori base è positivo, significa che aumentare il flusso su quell'arco ridurrebbe il costo totale del flusso. Al contrario, se il coefficiente di costo ridotto di un arco fuori base è nullo, significa che aumentare il flusso su quell'arco non avrebbe alcun effetto sul costo totale del flusso.

In pratica, questo risultato è importante perché suggerisce che se i coefficienti di costo ridotto sono tutti non negativi, si può essere certi che la soluzione corrente di base è ottimale e non ci sono flussi migliori con un costo inferiore. In altre parole, non c'è alcun margine di miglioramento per la soluzione di base corrente e quindi non c'è bisogno di eseguire ulteriori iterazioni dell'algoritmo di programmazione lineare a costo minimo.

\subsection{How to calcolare?}
Per applicare l'algoritmo del simplesso a costo minimo per i problemi di flusso, è necessario avere una soluzione di base iniziale. Una possibile strategia per ottenere una soluzione di base iniziale è quella di selezionare un albero di supporto del grafo residuo e utilizzare gli archi dell'albero come base iniziale. 

Per selezionare l'albero di supporto, si può partire da una soluzione di flusso nulla e iterativamente aggiungere archi all'albero di supporto finché non si raggiunge una soluzione di base ammissibile. 

Per aggiungere un arco fuori dalla base all'albero di supporto, si può seguire la procedura seguente:

\begin{enumerate}
\item Aggiungere l'arco fuori dalla base all'albero di supporto corrispondente alla base attuale.
\item Considerare l'unico ciclo che si forma con tale aggiunta.
\item Fissare come verso del ciclo quello dell'arco fuori dalla base.
\item Calcolare il coefficiente di costo ridotto sommando tra loro tutti i costi relativi agli archi attraversati dal ciclo nel loro stesso verso e sottraendo al risultato i costi degli archi attraversati dal ciclo in senso opposto.
\end{enumerate}

Il coefficiente di costo ridotto così calcolato indica la variazione del costo totale del flusso se l'arco fuori dalla base viene aggiunto al flusso corrente, mantenendo il flusso su tutti gli altri archi della base costante. 

Se il coefficiente di costo ridotto dell'arco fuori dalla base è negativo, è conveniente aumentare il flusso su tale arco per ridurre il costo totale del flusso. In tal caso, si può utilizzare l'algoritmo del simplesso a costo minimo per determinare la nuova soluzione di base ammissibile che minimizza il costo totale del flusso, partendo dalla soluzione di base attuale e utilizzando l'arco fuori dalla base appena aggiunto come variabile in entrata.

\subsection{Condizione di illimitatezza}
La condizione di illimitatezza nel problema di flusso a costo minimo si verifica quando l'aggiunta di un arco fuori dalla base con coefficiente di costo ridotto negativo crea un ciclo orientato nel grafo residuo. Questo significa che è possibile aumentare il flusso lungo il ciclo orientato senza limiti, riducendo continuamente il costo totale del flusso.

In altre parole, se si ha un ciclo orientato nel grafo residuo, è possibile aumentare il flusso su tale ciclo senza limiti, riducendo il costo totale del flusso ad ogni passo. Questo comporta che non esiste una soluzione di costo minimo nel problema di flusso a costo minimo, poiché il costo del flusso può essere ridotto all'infinito attraverso l'aumento del flusso lungo il ciclo orientato.

Pertanto, se si verifica questa condizione, il problema di flusso a costo minimo ha obiettivo illimitato e non esiste una soluzione di costo minimo finito. In tal caso, si può concludere che il problema non ha soluzione, oppure che la soluzione del problema richiede una restrizione addizionale al modello originale per evitare la presenza di cicli orientati.

\subsection{Cambio di base}
Se non sono soddisfatte le condizioni di ottimalità e illimitatezza procedo con un cambio di base:
\begin{itemize}
  \item scelgo un arco fuori base da inserire nella nuova base
  \item scelgo un arco in base da far uscire
\end{itemize}

Per scegliere gi archi da inserire in base, si deve scegliere quelli con coefficiente di costo ridotto negativo.


\subsection{Aggiornamento del flusso(metodo del ciclo di aumento)}
\begin{enumerate}
\item Aggiungere l'arco che si vuole far entrare nella base.
\item Si formerà un ciclo, che viene orientato secondo il verso di questo arco.
\item Si porta il flusso lungo questo arco a $\Delta$.
\item Si incrementa di $\Delta$ il flusso lungo gli archi del ciclo attraversati secondo il proprio verso.
\item Si decrementa di $\Delta$ il flusso lungo gli archi attraversati in verso opposto al proprio.
\end{enumerate}

In questo modo, il flusso totale viene aumentato di $\Delta$ lungo il ciclo e, contemporaneamente, il costo totale viene ridotto del valore corrispondente al costo del ciclo, che è calcolato come la somma dei costi degli archi attraversati secondo il proprio verso meno i costi degli archi attraversati in senso opposto. Il metodo del ciclo di aumento viene ripetuto finché non si raggiunge una soluzione di flusso a costo minimo.

Questo procedimento sfrutta il fatto che, se esiste una soluzione di flusso a costo minimo, allora deve esistere un insieme di archi (la base) tale che il flusso lungo questi archi sia massimo e che tutti gli archi al di fuori di questa base abbiano coefficienti di costo ridotto non negativi. L'aggiunta di un arco fuori dalla base con coefficiente di costo ridotto negativo crea un ciclo orientato nel grafo residuo, che può essere utilizzato per ridurre il costo totale del flusso e aumentare il flusso totale lungo il ciclo, portando così ad una soluzione migliore.


\subsection{Algoritmo del simplesso su rete}
Una volta completate le operazioni relative alla base e definita una nuova base, 
l'algoritmo del simplesso su rete ripete le operazioni sulla nuova base, iterando
questo procedimento fino a soddisfare le condiioni di ottimalità oppure quelle 
di illimitatezza.

Per utilizzare l'algoritmo del simplesso su rete devo avere una base da cui partire
e si utilizza un metodo formato da due fasi per ottenere la base di partenza ammissibile.


\textbf{Fase 1:}
\begin{enumerate}
\item Si aggiunge un nuovo nodo sorgente $s'$ e si collega ad ogni nodo sorgente esistente tramite un arco di capacità infinita e costo nullo.
\item Si aggiunge un nuovo nodo destinazione $t'$ e si collega ad ogni nodo destinazione esistente tramite un arco di capacità infinita e costo nullo.
\item Si risolve il problema di flusso massimo da $s'$ a $t'$ utilizzando un algoritmo come l'algoritmo di Ford-Fulkerson.
\item Se il flusso massimo ottenuto è inferiore alla somma delle capacità dei nodi sorgente originali, allora non esiste una base ammissibile per il problema e l'algoritmo termina.
\end{enumerate}

\textbf{Fase 2:}
\begin{enumerate}
\item Si costruisce l'albero di supporto dell'ultimo flusso trovato nella fase 1.
\item Se l'albero di supporto non contiene tutti i nodi, allora l'algoritmo termina in quanto non esiste una base ammissibile per il problema.
\item Se l'albero di supporto contiene tutti i nodi, allora si utilizza l'albero di supporto per costruire una base ammissibile per il problema utilizzando l'algoritmo del simplesso su rete.
\end{enumerate}

L'idea alla base di questo algoritmo è di utilizzare la fase 1 per verificare l'esistenza di una soluzione ammissibile e la fase 2 per costruire una base ammissibile utilizzando l'albero di supporto del flusso massimo trovato nella fase 1.

\subsection{Algoritmo del simplesso su rete(archi con vincoli di flusso)}
\begin{enumerate}
\item Partendo dalla base ammissibile $B_0$, calcolare il flusso massimo $f_{B_0}$ che può passare attraverso di essa risolvendo il problema di flusso massimo su s-t cut.

\item Calcolare i coefficienti di costo ridotto $c_e' = c_e - \pi_{h(e)} + \pi_{t(e)}$ per ogni arco $e$ fuori dalla base attuale, dove $\pi$ è il potenziale associato al nodo.

\item Se tutti i coefficienti di costo ridotto sono non negativi, allora la soluzione corrente è ottima. Altrimenti, scegliere un arco $e$ con coefficiente di costo ridotto negativo e creare un ciclo orientato con tale arco.

\item Determinare il flusso massimo $f_e$ che può passare attraverso l'arco $e$ selezionato. Se $f_e = 0$, andare al passo 3 e selezionare un nuovo arco con coefficiente di costo ridotto negativo. Altrimenti, se il flusso massimo $f_e$ è minore del flusso $f_{B_0}$ passante attraverso la base attuale, allora la soluzione ottima è infinita (condizione di illimitatezza). Altrimenti, se il flusso $f_e$ è uguale a $f_{B_0}$, allora il flusso rimane ammissibile e si va al passo 6. Altrimenti, se $f_e > f_{B_0}$, si va al passo 5.

\item Aggiornare il flusso $f_{B_0}$ con $f_e$ e ottenere la nuova base ammissibile $B_1$. Tornare al passo 2.

\item Utilizzando il ciclo orientato individuato al passo 3, aggiornare il flusso lungo gli archi del ciclo orientato come segue:
\begin{itemize}
\item Se un arco del ciclo orientato è diretto nello stesso verso dell'arco selezionato $e$, decrementare il flusso su tale arco di $f_e$.
\item Se un arco del ciclo orientato è diretto in senso opposto all'arco selezionato $e$, incrementare il flusso su tale arco di $f_e$.
\end{itemize}
Tornare al passo 2.
\end{enumerate}

\section{Problema di flusso massimo}

Il problema di flusso massimo consiste nel trovare il flusso di massimo valore che può essere trasmesso attraverso una rete di flusso. In altre parole, data una rete di flusso con una sorgente e un pozzo, il problema di flusso massimo cerca di trovare il flusso massimo che può essere inviato dalla sorgente al pozzo attraverso la rete, rispettando i vincoli di capacità degli archi.

Il flusso in una rete di flusso rappresenta la quantità di materiale (ad esempio liquido, gas, dati) che viene trasmesso attraverso la rete. Il flusso viene misurato in unità di flusso (ad esempio litri al secondo), e ogni arco nella rete ha una capacità massima di flusso che può trasportare. Il flusso inviato dalla sorgente deve essere uguale al flusso ricevuto dal pozzo, e il flusso attraverso ogni arco non può superare la sua capacità massima.

Il problema di flusso massimo può essere risolto utilizzando l'algoritmo di Ford-Fulkerson, che prevede l'incremento del flusso lungo un cammino aumentante nella rete fino a quando non è possibile trovare ulteriori cammini aumentanti. Un cammino aumentante è un percorso dalla sorgente al pozzo attraverso la rete in cui il flusso attraverso ogni arco non ha ancora raggiunto la capacità massima.

L'algoritmo di Ford-Fulkerson può essere implementato utilizzando l'algoritmo del simplesso sul grafo residuo della rete. Il grafo residuo è un grafo diretto che rappresenta le capacità residue degli archi nella rete, dopo che un certo flusso è stato inviato attraverso la rete. L'algoritmo del simplesso viene utilizzato per trovare il cammino aumentante di capacità massima nel grafo residuo e incrementare il flusso lungo tale cammino.

L'algoritmo di Ford-Fulkerson termina quando non è più possibile trovare cammini aumentanti nel grafo residuo. In quel momento, il flusso trovato è il flusso di massimo valore che può essere trasmesso attraverso la rete.

\subsection{Tagli nella rete}

Un taglio $C$ in una rete di flusso è una partizione del set di nodi $V$ in due sottoinsiemi disgiunti $S$ e $T=V\setminus S$ tali che la sorgente $s$ appartiene a $S$ e il pozzo $t$ appartiene a $T$. Il taglio attraversato da un flusso è il taglio che è attraversato da almeno un arco del flusso. Il valore del taglio $C=(S,T)$ è la somma dei flussi che attraversano tutti gli archi tra $S$ e $T$. In altre parole:

$$
val(C) = \sum_{i\in S, j\in T} f_{ij}
$$

dove $f_{ij}$ rappresenta il flusso lungo l'arco $(i,j)$.

Un taglio minimo è un taglio attraverso il quale passa il minimo valore di flusso possibile. In altre parole, è un taglio che ha il valore minimo tra tutti i possibili tagli della rete.

\subsection{Algoritmo di Ford-Fulkerson}

\begin{algorithm}[H]
\caption{Algoritmo di Ford-Fulkerson}
\begin{algorithmic}[1]
\Procedure{Ford-Fulkerson}{$G, s, t$}
\State Inizializza il flusso $f(u, v) = 0$ per ogni arco $(u,v)\in G$
\While{esiste un cammino aumentante $p$ in $G_f$}
    \State Trova la capacità residua minima $c_f(p)$ dell'arco in $p$
    \ForAll{gli archi $(u,v)\in p$}
        \State Aggiorna il flusso: $f(u,v) \gets f(u,v) + c_f(p)$
        \State Aggiorna il flusso inverso: $f(v,u) \gets f(v,u) - c_f(p)$
    \EndFor
\EndWhile
\State \textbf{return} $f$
\EndProcedure
\end{algorithmic}
\end{algorithm}

Nota: $G_f$ indica il grafo residuo corrispondente al flusso $f$. Il cammino aumentante $p$ è un cammino dal nodo sorgente $s$ al nodo destinazione $t$ in $G_f$ tale che la capacità residua $c_f(p)$ sia strettamente positiva per ogni arco in $p$.

\subsection{Procedura di etichettatura}

\begin{enumerate}
  \item Inizializzare il flusso $f_{u,v} = 0$ per ogni arco $(u,v)$ nel grafo.
  \item Scegliere un cammino aumentante dal nodo di sorgente $s$ al nodo di destinazione $t$ utilizzando la procedura di ricerca di cammino aumentante come l'algoritmo di Ford-Fulkerson.
  \item Se non ci sono cammini aumentanti, terminare: il flusso corrente è massimo.
  \item Calcolare la capacità residua minima lungo il cammino aumentante trovato: 
  $$c_f(p) = \min_{(u,v) \in p} (c(u,v) - f(u,v))$$
  dove $p$ è il cammino aumentante.
  \item Aggiornare il flusso lungo il cammino aumentante: 
  $$f(u,v) \leftarrow f(u,v) + c_f(p)$$ per ogni arco $(u,v)$ in $p$.
  \item Aggiornare il flusso lungo gli archi di ritorno del cammino aumentante: 
  $$f(v,u) \leftarrow f(v,u) - c_f(p)$$ per ogni arco $(v,u)$ in $p$.
  \item Ritornare al passo 2.
\end{enumerate}

    \chapter{Assegnamento}
L'assegnamento è un problema di ottimizzazione che consiste nel trovare la soluzione ottima di un insieme di assegnamenti tra un gruppo di lavoratori e un gruppo di mansioni, considerando i costi associati ad ogni possibile assegnamento. In altre parole, l'assegnamento cerca di minimizzare il costo totale dell'assegnamento dei lavoratori alle mansioni, soddisfacendo allo stesso tempo i vincoli di assegnamento.

Per questo tipo di task si utilizzano delle matrici che rappresentano i costi,
ecco un esempio di matrice dei costi per l'assegnamento di lavoratori a mansioni:

\begin{equation}
\begin{bmatrix}
 & A_1 & A_2 & A_3 & A_4 & A_5 \\
B_1 & c_{11} & c_{12} & c_{13} & c_{14} & c_{15} \\
B_2 & c_{21} & c_{22} & c_{23} & c_{24} & c_{25} \\
B_3 & c_{31} & c_{32} & c_{33} & c_{34} & c_{35} \\
B_4 & c_{41} & c_{42} & c_{43} & c_{44} & c_{45} \\
B_5 & c_{51} & c_{52} & c_{53} & c_{54} & c_{55}
\end{bmatrix}
\end{equation}

In questa matrice, $c_{ij}$ rappresenta il costo associato all'assegnamento del lavoratore $B_i$ alla mansione $A_j$.

\section{Algoritmo ungherese}
L'algoritmo ungherese, noto anche come algoritmo di Kuhn-Munkres, è un algoritmo per la soluzione del problema dell'assegnamento, ovvero il problema di assegnare N lavoratori a N mansioni, minimizzando i costi totali dell'assegnamento. L'algoritmo utilizza il concetto di cammino alternante e cammino aumentante per trovare l'assegnamento ottimale.
\subsection{Ridurre la matrice dei costi}

Per ridurre la matrice dei costi in un problema di assegnamento, puoi seguire i seguenti passi:
\begin{enumerate}
  
\item Sottrarre il minimo valore in ogni riga dalla riga stessa.
\item Sottrarre il minimo valore in ogni colonna dalla colonna stessa.
\item Coprire la matrice con il minor numero di linee orizzontali e verticali in modo che ogni cella sia coperta da almeno una linea.
\item Calcolare il costo minimo delle celle non coperte.
\item Sottrarre il costo minimo alle celle coperte dalle linee orizzontali e aggiungere il costo minimo alle celle coperte dalle linee verticali.
\item Ripetere i passi 3-5 finché non è possibile coprire la matrice con meno linee del passo precedente.

\end{enumerate}
La matrice ridotta avrà gli stessi elementi della matrice originale ma con alcune righe o colonne che contengono solamente zeri. 

Ecco un esempio di matrice dei costi ridotta:

$$
\begin{pmatrix}
0 & 1 & 5 & 8 & 1 \\
2 & 5 & 0 & 5 & 2 \\
1 & 0 & 3 & 6 & 7 \\
8 & 5 & 7 & 0 & 3 \\
3 & 3 & 2 & 1 & 0
\end{pmatrix}
\rightarrow
\begin{pmatrix}
0 & 0 & 4 & 7 & 0 \\
1 & 4 & 0 & 4 & 1 \\
0 & 0 & 2 & 5 & 6 \\
5 & 2 & 4 & 0 & 1 \\
2 & 2 & 1 & 0 & 0
\end{pmatrix}
\rightarrow
\begin{pmatrix}
x & 0 & 4 & 7 & 0 \\
0 & 4 & 0 & 4 & 1 \\
0 & 0 & 2 & 5 & 6 \\
5 & 2 & 4 & 0 & 1 \\
2 & 2 & 1 & 0 & 0
\end{pmatrix}
\rightarrow
\begin{pmatrix}
x & 0 & 4 & 7 & 0 \\
0 & 4 & 0 & 4 & 1 \\
0 & 0 & 2 & 5 & 6 \\
5 & 2 & 4 & 0 & 1 \\
0 & 0 & 0 & 0 & x
\end{pmatrix}
$$

Nella matrice ridotta, le righe 1, 2 e 4 e le colonne 2, 4 e 5 contengono solamente zeri. Il costo complessivo della soluzione ottima è pari alla somma dei valori nella matrice originale corrispondenti alle celle che contengono un "x".

\begin{algorithm}[H]
\SetAlgoLined
\KwResult{Assegnamento ottimale di lavoratori a mansioni}
Riduci la matrice dei costi;
Trova una soluzione di base ammissibile;
\While{non si raggiunge la soluzione ottimale}{
Trova un cammino aumentante;
Incrementa il flusso lungo il cammino aumentante;
Decrementa il flusso lungo gli archi corrispondenti nel cammino aumentante;
}
\caption{Algoritmo ungherese}
\end{algorithm}

    \chapter{Branch and Bound}

Questa tecnica si utilizza quando si devono risolvere problemi di programmazione lineare intera (PLI) o di programmazione mista intera (PMI). Il suo funzionamento si basa su una strategia di divide et impera, che consiste nel suddividere il problema in sotto-problemi più piccoli e risolvibili più facilmente.

\section{Compontenti dell'algoritmo}

\subsection{Upper Bound}
L'upper bound rappresenta un limite superiore per il valore della soluzione ottima. Viene utilizzato per stabilire un valore di riferimento iniziale per valutare le soluzioni parziali e confrontarle con la soluzione ottima corrente. Durante l'esecuzione dell'algoritmo, se viene trovata una soluzione parziale che supera il limite superiore corrente, il sottoproblema associato a quella soluzione può essere scartato, poiché non può migliorare la soluzione ottima trovata finora. L'obiettivo è ridurre il limite superiore durante l'esecuzione dell'algoritmo, fino a raggiungere il valore della soluzione ottima.

\subsection{Lower Bound}
Il lower bound rappresenta un limite inferiore per il valore della soluzione ottima. Viene calcolato per ogni sottoproblema generato durante l'esecuzione dell'algoritmo. Il lower bound può essere ottenuto utilizzando euristiche o algoritmi di approssimazione per stimare il valore minimo che può essere ottenuto dalla soluzione ottima in quel sottoproblema. Se il lower bound di un sottoproblema è maggiore del limite superiore corrente, il sottoproblema e il suo sottoalbero di soluzioni possono essere eliminati dall'esplorazione, poiché non possono migliorare la soluzione ottima trovata finora.

\subsection{Valore Obiettivo}
Il valore obiettivo è il valore associato alla soluzione ottima del problema di ottimizzazione. L'obiettivo dell'algoritmo Branch and Bound è trovare una soluzione che abbia il valore obiettivo migliore possibile. Durante l'esecuzione dell'algoritmo, se viene trovata una soluzione parziale con un valore obiettivo migliore del limite superiore corrente, il limite superiore viene aggiornato con il nuovo valore obiettivo. L'obiettivo finale è raggiungere la soluzione ottima con il valore obiettivo massimo o minimo, a seconda del tipo di problema di ottimizzazione.

\section{Algoritmo}
\begin{algorithm}[H]
\SetAlgoLined
\KwIn{Problema di ottimizzazione}
\KwOut{Soluzione ottima}

Inizializza una soluzione parziale vuota;
Inizializza un limite superiore iniziale ad un valore molto alto;

\While{Non sono stati esplorati tutti i sottoproblemi}{
Genera un nuovo sottoproblema più piccolo;
Calcola un limite inferiore per il sottoproblema;

\If{Limite inferiore $>$ Limite superiore}{
Non esplorare ulteriormente il sottoproblema;
}
\Else{
Esplora il sottoproblema generando ulteriori soluzioni parziali;
\If{Soluzione parziale completa}{
  \If{Valore obiettivo $<$ Limite superiore}{
    Aggiorna il limite superiore con il valore obiettivo\;
  }
}
}
}

\Return{Soluzione ottima corrispondente al limite superiore migliore};
\caption{Algoritmo Branch and Bound}
\end{algorithm}

    \chapter{Problema KNAPSACK}
Il problema dello zaino (in inglese, knapsack problem) è un problema di ottimizzazione combinatoria. Si supponga di avere uno zaino con una capacità massima e una serie di oggetti, ognuno dei quali ha un determinato valore e un certo peso. Lo scopo è riempire lo zaino con gli oggetti in modo da massimizzare il valore totale degli oggetti, rispettando la capacità massima dello zaino.

Formalmente, il problema può essere definito come segue: dati n oggetti con un valore $v_i$ e un peso $w_i$ per $i = 1,...,n$ e una capacità massima $W$ dello zaino, trovare una combinazione di oggetti che massimizzi il valore totale degli oggetti, rispettando la capacità massima dello zaino. In altre parole, si cerca di trovare una soluzione x, dove $x_i \in \{0,1\}$ rappresenta se l'oggetto i è incluso o meno nello zaino, tale che

$$\sum_{i=1}^n w_i x_i \le W$$
$$\max \sum_{i=1}^n v_i x_i$$

dove $\sum_{i=1}^n w_i x_i \le W$ rappresenta il vincolo sulla capacità dello zaino.

Il problema dello zaino è un problema di ottimizzazione combinatoria NP-completo, il che significa che non esiste un algoritmo che possa risolvere il problema in un tempo polinomiale in base alla dimensione dell'input. Tuttavia, esistono algoritmi efficienti per risolvere il problema in alcuni casi particolari, come quando tutti i pesi e i valori degli oggetti sono interi positivi.

\section{Brach and bound per KNAPSACK}
L'algoritmo di branch and bound applicato al problema dello zaino (knapsack problem) prevede i seguenti passi:
\begin{enumerate}
  
\item Inizializzazione: si parte dalla soluzione vuota e si calcola l'upper bound (UB) iniziale come la soluzione ottenuta con il rilassamento continuo del problema.
\item Branching: si seleziona un elemento non ancora selezionato e si generano due sotto-problemi, uno in cui l'elemento viene inserito nello zaino e uno in cui non viene inserito. Per ogni sotto-problema, si calcola l'upper bound e si scarta il sotto-problema se l'upper bound è minore della migliore soluzione trovata finora.
\item Selezione del sotto-problema: si seleziona il sotto-problema con l'upper bound più alto tra quelli non ancora scartati e si ripete il processo dal passo 2.
\item Terminazione: l'algoritmo termina quando tutti i sotto-problemi sono stati esplorati o quando non ci sono più sotto-problemi con un upper bound maggiore della migliore soluzione trovata finora.

\end{enumerate}
Durante l'esplorazione dell'albero dei sotto-problemi, l'upper bound viene calcolato come la somma dei valori degli elementi già selezionati e del valore massimo che si può ancora ottenere con gli elementi rimanenti, tenendo conto del limite di peso dello zaino.

La soluzione ottimale viene ottenuta quando l'algoritmo termina e restituisce la migliore soluzione trovata finora.

\begin{algorithm}[H]
\SetAlgoLined
\KwIn{Array di $n$ oggetti con peso $p_i$ e valore $v_i$, capacità dello zaino $W$}
\KwOut{Valore massimo che si può ottenere riempiendo lo zaino}
\textbf{inizializzazione:} $u \gets 0, LB \gets 0, x \gets \vec{0}, j \gets 1, z \gets 0$ \\
$\mathrm{Heap}:$ tutti i nodi sono presenti nel heap con $LB$ come chiave \\
\While{Heap non vuoto}{
    $N \gets$ nodo con il valore minimo di $LB$ \\
    \If{$LB \geq u$}{
        \textbf{return} $u$
    }
    $i \gets$ livello di profondità del nodo $N$ \\
    \If{$i>n$}{
        \textbf{continue}
    }
    \If{$z+v_i \leq u$}{
        \textbf{continue}
    }
    \textbf{crea} il figlio sinistro $S$ del nodo $N$ mettendo nell'array $x$ il valore 0 per l'oggetto $i$ \\
    \textbf{crea} il figlio destro $D$ del nodo $N$ mettendo nell'array $x$ il valore 1 per l'oggetto $i$ \\
    \textbf{calcola} il lower bound $LB_S$ del figlio sinistro $S$ \\
    \textbf{calcola} il lower bound $LB_D$ del figlio destro $D$ \\
    \If{$LB_S < u$}{
        \textbf{inserisci} $S$ nel heap con $LB_S$ come chiave
    }
    \If{$LB_D < u$}{
        \textbf{inserisci} $D$ nel heap con $LB_D$ come chiave
    }
}
\textbf{return} $u$
\caption{Algoritmo Branch and Bound per il problema dello zaino (Knapsack)}
\end{algorithm}

    \chapter{Programmazione dinamica}

La programmazione dinamica \`e applicabile a problemi che rispettano:
\begin{itemize}
  \item Il problema pu\`o essere suddiviso in sottoproblemi pi\`u piccoli.
  \item In ogni sotto blocco $k$, con $k = 1, \dots, n$, ci si trova in uno degli stati
        possibili $S_k$.
  \item In ogni blocco si deve prendere una decisione $d_k$ che appartiene al dominio
        delle decisioni $D_k$.
\end{itemize}

In ogni blocco $k$ si ha che la funzione obiettivo $f_k$ \'e $u(d_k, s_k)$

Se in un momento mi trovo nel blocco $k$ con decisione $d_k$ e stato $s_k$, posso passare allo stato successivo
$s_{k+1} = t(d_k, s_k)$ dove la funzione di transizione $t$ \'e definita come funzione di transizione.




\section{Il principio di ottimalità}
Il principio di ottimalità è un concetto fondamentale della programmazione dinamica. Esso afferma che una soluzione ottima a un problema di ottimizzazione globale può essere costruita attraverso le soluzioni ottime dei suoi sotto-problemi. In altre parole, se un problema può essere suddiviso in sotto-problemi più piccoli, la soluzione ottima del problema globale può essere ottenuta combinando le soluzioni ottime dei suoi sotto-problemi.

Questo principio si applica quando si hanno problemi in cui la soluzione ottima di una istanza del problema contiene al suo interno la soluzione ottima di sotto-istanze del problema stesso. In questo caso, la soluzione del problema può essere ottenuta risolvendo le sotto-istanze e combinando le loro soluzioni in modo opportuno.

L'utilizzo del principio di ottimalità permette di evitare di risolvere più volte lo stesso sotto-problema, riducendo così il tempo di calcolo e aumentando l'efficienza dell'algoritmo di programmazione dinamica.


\section{Programmazione dinamica per il problema dello zaino}


Si consideri il problema dello zaino in cui si ha a disposizione uno zaino di capacità $C$ e un insieme di $n$ oggetti. Ogni oggetto $i$ ha un peso $p_i$ e un valore $v_i$. Si vuole trovare la combinazione di oggetti che massimizza il valore totale, rispettando la capacità dello zaino.

Definiamo $K(i, w)$ la soluzione ottima del problema dello zaino utilizzando i primi $i$ oggetti e uno zaino di capacità $w$. Il problema può essere risolto tramite programmazione dinamica utilizzando il principio di ottimalità.

Il principio di ottimalità afferma che una soluzione ottima al problema dello zaino che considera i primi $i$ oggetti, è ottenuta considerando o l'oggetto $i$ o meno. Quindi, la soluzione ottima può essere ottenuta confrontando il valore massimo che si può ottenere considerando l'oggetto $i$ (e utilizzando uno zaino di capacità $w-p_i$) con il valore massimo che si può ottenere senza considerare l'oggetto $i$ (e utilizzando uno zaino di capacità $w$). Formalmente:

\[
K(i, w) = 
\begin{cases} 
0 & i = 0 \text{ o } w = 0 \\
K(i-1, w) & p_i > w \\
\max \{ K(i-1, w), K(i-1, w-p_i) + v_i \} & \text{altrimenti}
\end{cases}
\]

La soluzione al problema dello zaino può essere trovata calcolando $K(n, C)$.

Inoltre, è possibile utilizzare la programmazione dinamica per trovare la combinazione di oggetti che massimizza il valore totale. Dopo aver calcolato la matrice $K(i,w)$, si può risalire ai singoli oggetti utilizzati nella soluzione ottima tramite la seguente procedura:

\begin{algorithm}[H]
\SetAlgoLined
\KwResult{Oggetti utilizzati nella soluzione ottima}
$i \leftarrow n$\;
$w \leftarrow C$\;
\While{$i > 0$}{
  \If{$K(i,w) \neq K(i-1,w)$}{
    Utilizza l'oggetto $i$\;
    $w \leftarrow w - p_i$\;
  }
  $i \leftarrow i - 1$\;
}
\caption{Procedura per ottenere gli oggetti utilizzati nella soluzione ottima}
\end{algorithm}


\section{Che cos'è l'algoritmo valore ottimo?}

L'algoritmo valore ottimo (in inglese \textit{Value Iteration}) è un algoritmo di programmazione dinamica utilizzato per trovare la politica ottima in un processo decisionale di Markov a tempo discreto (MDP). L'algoritmo è basato sull'iterazione dei valori e utilizza una procedura di backup dei valori per aggiornare il valore di ogni stato.

L'idea alla base dell'algoritmo valore ottimo è quella di trovare la funzione valore ottimo $V^*(s)$ di ogni stato $s$ del MDP. Questa funzione indica il valore atteso della ricompensa totale che si può ottenere partendo dallo stato $s$ e seguendo la politica ottima. L'algoritmo itera la stima della funzione valore ottimo fino a raggiungere una convergenza. Alla fine di ogni iterazione, l'algoritmo aggiorna la funzione valore ottimo di ogni stato sulla base del valore atteso dei suoi successori.

L'algoritmo valore ottimo è un metodo molto efficace per risolvere MDPs di grandi dimensioni, tuttavia richiede la conoscenza completa del modello del sistema, ovvero delle probabilità di transizione e delle ricompense associate ad ogni transizione.


\subsection{Applicato a KNAPSACK}

\begin{algorithm}[H]
\SetAlgoLined
\KwIn{Un insieme di $n$ oggetti, dove ogni oggetto $i$ ha un valore $v_i$ e un peso $w_i$, e una capacità massima dello zaino $W$}
\KwOut{Il valore massimo che può essere ottenuto con una combinazione di oggetti che non superi la capacità $W$}

$A \gets$ array $n \times (W+1)$ inizializzato a 0\;

\For{$i \gets 1$ \KwTo $n$}{
    \For{$w \gets 1$ \KwTo $W$}{
        \eIf{$w_i \leq w$}{
            $A[i, w] \gets \max\{A[i-1, w], A[i-1, w-w_i]+v_i\}$\;
        }{
            $A[i, w] \gets A[i-1, w]$\;
        }
    }
}
\Return{$A[n, W]$}
\caption{Algoritmo valore ottimo per il problema dello zaino}
\label{alg:valore-ottimo-zaino}
\end{algorithm}

In questo algoritmo, si utilizza una matrice $A$ di dimensioni $n \times (W+1)$ per memorizzare i valori ottimi per tutti i sottoproblemi del problema dello zaino, ovvero il valore massimo che può essere ottenuto con una combinazione di oggetti che non superi una capacità $w \leq W$ e che includa solo i primi $i$ oggetti.

L'algoritmo utilizza una doppia iterazione sui primi $i$ oggetti e sulle capacità $w$. Se l'oggetto $i$ ha peso $w_i$ minore o uguale alla capacità $w$ considerata, allora è possibile scegliere se includere o meno l'oggetto $i$ nella combinazione. Se si decide di includerlo, il valore massimo ottenibile è dato dalla somma del valore dell'oggetto $i$ e del valore massimo ottenibile con i primi $i-1$ oggetti e una capacità residua di $w-w_i$. Altrimenti, il valore massimo ottenibile è dato dal valore massimo ottenibile con i primi $i-1$ oggetti e la capacità $w$ considerata.

Alla fine delle iterazioni, il valore ottimo per il problema originale, ovvero la combinazione di oggetti con valore massimo che non superi la capacità $W$, è dato dall'elemento $A[n, W]$ della matrice.

\section{Problema della schedulazione}

Il \textbf{problema della schedulazione} consiste nell'assegnare un insieme di \textit{n} attività da eseguire, ognuna delle quali richiede un certo tempo di elaborazione, a un insieme di \textit{m} risorse, ciascuna delle quali ha una certa disponibilità. L'obiettivo è minimizzare il tempo totale di completamento delle attività.

Formalmente, il problema può essere definito come segue:

Dati:
\begin{itemize}
    \item Un insieme di \textit{n} attività $A = \{a_1, a_2, ..., a_n\}$.
    \item Un insieme di \textit{m} risorse $R = \{r_1, r_2, ..., r_m\}$.
    \item Un tempo di elaborazione $p_{ij}$ per eseguire l'attività $a_i$ sulla risorsa $r_j$.
    \item Una disponibilità $d_j$ per ogni risorsa $r_j$.
\end{itemize}
Obiettivo: Trovare un assegnamento delle attività alle risorse che minimizzi il tempo totale di completamento.


Il problema della schedulazione può essere formulato in diverse varianti, come ad esempio il problema della schedulazione su una singola macchina o il problema della schedulazione su più macchine. In generale, il problema è considerato \textit{NP-hard}, il che significa che non esiste un algoritmo efficiente in grado di risolverlo in tempo polinomiale per tutti i casi.

    \chapter{Problemi di ottimizzazione}
I problemi di ottimizzazione sono un tipo di problemi matematici in cui si cerca di trovare il valore massimo o minimo di una funzione, nota come funzione obiettivo, in presenza di una serie di vincoli. Questi vincoli possono essere di natura diversa, ad esempio possono essere delle equazioni o delle disuguaglianze che limitano l'insieme di soluzioni ammissibili per il problema.

Ogni istanza \'e rappresentata come:
\begin{eqnarray}
  f: S \rightarrow R
\end{eqnarray}
Dove $S$ \'e l'insieme delle soluzioni ammissibili e $R$ \'e l'insieme dei valori che la funzione obiettivo pu\'o assumere.


\section{La classe P}

Dato un problema di ottimizzazione $R$, diciamo che questo appartiene
alla classe P se esiste un algoritmo $A$ di complessit\'a polinomiale che
lo risolve.

Alcuni esempi di problemi in P sono:
\begin{itemize}
  \item \textbf{SHORT PATH}: dato un grafo $G = (V, E)$, un nodo $s \in V$ e un nodo $t \in V$, trovare il cammino pi\'u breve da $s$ a $t$.
  \item \textbf{MST}: dato un grafo $G = (V, E)$, trovare un albero di copertura minimo. 
\end{itemize}


\section{La classe NP}

La classe NP contiene tutti i problemi di ottimizzazione per i quali, 
nota la soluzione ottima, \'e possibile calcolarlo in tempo polinomiale.

Alcuni esempi di problemi in NP sono:
\begin{itemize}
  \item \textbf{CLIQUE}: dato un grafo $G = (V, E)$ e un intero $k$, trovare un sottoinsieme di $k$ nodi di $V$ che formano un grafo completo.
  \item \textbf{TSP}: dato un grafo $G = (V, E)$, trovare un ciclo hamiltoniano di costo minimo.
\end{itemize}


\section{La classe NP-completi}

Un problema $R$ \'e NP-completo se:
\begin{itemize}
  \item $R \in NP$
\item $\forall Q \in NP \exists $ riduzione polinomiale di $Q$ in $R$
\end{itemize}

Esempi di problemi NP-completi sono:
\begin{itemize}
  \item \textbf{CLIQUE}: dato un grafo $G = (V, E)$ e un intero $k$, trovare un sottoinsieme di $k$ nodi di $V$ che formano un grafo completo.
  \item \textbf{TSP}: dato un grafo $G = (V, E)$, trovare un ciclo hamiltoniano di costo minimo.
  \item \textbf{KNAPSACK}: dato un insieme di $n$ oggetti, ognuno con un peso $w_i$ e un valore $v_i$, e un intero $W$, trovare un sottoinsieme di oggetti che non superi il peso $W$ e che massimizzi il valore totale.
\end{itemize}




    
\end{document}
