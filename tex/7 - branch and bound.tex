\chapter{Branch and Bound}

Questa tecnica si utilizza quando si devono risolvere problemi di programmazione lineare intera (PLI) o di programmazione mista intera (PMI). Il suo funzionamento si basa su una strategia di divide et impera, che consiste nel suddividere il problema in sotto-problemi più piccoli e risolvibili più facilmente.

\section{Compontenti dell'algoritmo}

\subsection{Upper Bound}
L'upper bound rappresenta un limite superiore per il valore della soluzione ottima. Viene utilizzato per stabilire un valore di riferimento iniziale per valutare le soluzioni parziali e confrontarle con la soluzione ottima corrente. Durante l'esecuzione dell'algoritmo, se viene trovata una soluzione parziale che supera il limite superiore corrente, il sottoproblema associato a quella soluzione può essere scartato, poiché non può migliorare la soluzione ottima trovata finora. L'obiettivo è ridurre il limite superiore durante l'esecuzione dell'algoritmo, fino a raggiungere il valore della soluzione ottima.

\subsection{Lower Bound}
Il lower bound rappresenta un limite inferiore per il valore della soluzione ottima. Viene calcolato per ogni sottoproblema generato durante l'esecuzione dell'algoritmo. Il lower bound può essere ottenuto utilizzando euristiche o algoritmi di approssimazione per stimare il valore minimo che può essere ottenuto dalla soluzione ottima in quel sottoproblema. Se il lower bound di un sottoproblema è maggiore del limite superiore corrente, il sottoproblema e il suo sottoalbero di soluzioni possono essere eliminati dall'esplorazione, poiché non possono migliorare la soluzione ottima trovata finora.

\subsection{Valore Obiettivo}
Il valore obiettivo è il valore associato alla soluzione ottima del problema di ottimizzazione. L'obiettivo dell'algoritmo Branch and Bound è trovare una soluzione che abbia il valore obiettivo migliore possibile. Durante l'esecuzione dell'algoritmo, se viene trovata una soluzione parziale con un valore obiettivo migliore del limite superiore corrente, il limite superiore viene aggiornato con il nuovo valore obiettivo. L'obiettivo finale è raggiungere la soluzione ottima con il valore obiettivo massimo o minimo, a seconda del tipo di problema di ottimizzazione.

\section{Algoritmo}
\begin{algorithm}[H]
\SetAlgoLined
\KwIn{Problema di ottimizzazione}
\KwOut{Soluzione ottima}

Inizializza una soluzione parziale vuota;
Inizializza un limite superiore iniziale ad un valore molto alto;

\While{Non sono stati esplorati tutti i sottoproblemi}{
Genera un nuovo sottoproblema più piccolo;
Calcola un limite inferiore per il sottoproblema;

\If{Limite inferiore $>$ Limite superiore}{
Non esplorare ulteriormente il sottoproblema;
}
\Else{
Esplora il sottoproblema generando ulteriori soluzioni parziali;
\If{Soluzione parziale completa}{
  \If{Valore obiettivo $<$ Limite superiore}{
    Aggiorna il limite superiore con il valore obiettivo\;
  }
}
}
}

\Return{Soluzione ottima corrispondente al limite superiore migliore};
\caption{Algoritmo Branch and Bound}
\end{algorithm}
