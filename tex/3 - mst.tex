\chapter{Minimum Spanning Tree}

Un Minimum Spanning Tree (MST), tradotto in italiano come Albero di Supporto Minimo, è un albero di supporto di peso minimo in un grafo non diretto e connesso. L'MST è costituito da un sottoinsieme di archi del grafo che connette tutti i nodi in modo tale che il peso totale degli archi sia il più basso possibile.

Le caratteristiche principali degli MST sono:

\begin{enumerate}
    \item \textbf{Connessione}: Un MST deve connettere tutti i nodi del grafo, garantendo che non ci siano nodi isolati.
    \item \textbf{Aciclicità}: Un MST non deve contenere cicli, quindi non può avere archi che creano loop all'interno dell'albero.
    \item \textbf{Peso minimo}: Un MST ha il peso totale degli archi più basso possibile tra tutti gli alberi che soddisfano le prime due caratteristiche.
\end{enumerate}

Gli MST hanno diverse applicazioni pratiche, tra cui:

\begin{itemize}
    \item Reti di comunicazione: Un MST può essere utilizzato per collegare un insieme di punti in una rete di comunicazione minimizzando il costo totale dei collegamenti.
    \item Progettazione di reti stradali: Gli MST possono essere utilizzati per pianificare reti stradali efficienti, dove gli archi rappresentano le strade e il peso degli archi può essere la distanza o il tempo di percorrenza.
    \item Analisi dei dati: Gli MST possono essere utilizzati per individuare le relazioni più rilevanti o significative tra i dati, ad esempio nella visualizzazione delle relazioni tra punti di dati su una mappa.
\end{itemize}


\section{Algoritmo greedy per MST}
L'algoritmo greedy per MST (Minimum Spanning Tree) è un approccio basato sulla selezione di archi in base al loro peso. L'idea principale è quella di selezionare ripetutamente l'arco di peso minimo che collega un nodo dell'MST esistente a un nodo non ancora raggiunto, finché non viene creato un albero che connette tutti i nodi del grafo.



\begin{enumerate}
  
  \item Inizializzazione: Parti da un grafo non diretto e connesso $G = (V, E)$, dove $V$ rappresenta l'insieme dei nodi e $E$ rappresenta l'insieme degli archi. Crea un insieme vuoto MST che conterrà gli archi dell'albero di supporto minimo.
  \item Seleziona un nodo di partenza arbitrario. Questo può essere fatto in modo casuale o seguendo una strategia specifica, ad esempio selezionando il primo nodo dell'insieme dei nodi.
  \item Finché non sono stati raggiunti tutti i nodi:
    \begin{enumerate}
      \item Seleziona l'arco di peso minimo che collega un nodo nell'MST esistente a un nodo non ancora raggiunto. Questo arco deve essere selezionato tra gli archi che collegano il nodo raggiunto all'esterno dell'MST.
      \item Aggiungi l'arco selezionato all'MST.
      \item Marca il nodo raggiunto come "visitato" o "raggiunto".
    \end{enumerate}
  \item Alla fine del processo, l'MST conterrà tutti gli archi necessari per connettere tutti i nodi del grafo in modo che il peso totale dell'MST sia minimo.
\end{enumerate}
L'algoritmo greedy per MST può essere implementato utilizzando diverse strutture dati, come ad esempio una coda di priorità (heap) per selezionare l'arco di peso minimo in modo efficiente.

Ecco un esempio di implementazione dell'algoritmo greedy per MST in Python:

\begin{lstlisting}
  
def greedy_mst(graph):
    mst = []  # Insieme di archi dell'MST
    visited = set()  # Insieme di nodi visitati
    start_node = list(graph.keys())[0]  # Nodo di partenza arbitrario
    
    visited.add(start_node)
    
    while len(visited) < len(graph):
        min_weight = float('inf')
        min_edge = None
        
        # Scansiona gli archi collegati ai nodi visitati
        for node in visited:
            for neighbor, weight in graph[node]:
                if neighbor not in visited and weight < min_weight:
                    min_weight = weight
                    min_edge = (node, neighbor)
        
        if min_edge:
            mst.append(min_edge)
            visited.add(min_edge[1])
    
    return mst
\end{lstlisting}

Questo è solo un esempio di implementazione e può variare a seconda delle specifiche del problema. Assicurati di adattare l'algoritmo in base alle tue esigenze specifiche.

\subsection{Correttezza algoritmo}


La complessità dell'algoritmo greedy per MST dipende dalla rappresentazione del grafo e dalla struttura dati utilizzata.

Assumendo che il grafo sia rappresentato come una lista di adiacenza, con $n$ nodi e $m$ archi, e che si utilizzi una coda di priorità (heap) per selezionare l'arco di peso minimo, la complessità dell'algoritmo è:

\begin{itemize}
    \item Inizializzazione: $O(n)$
    \item Ciclo principale: $O(m \log m)$
\end{itemize}

All'interno del ciclo principale, l'estrazione dell'arco di peso minimo richiede un'operazione di estrazione minima dalla coda di priorità, che ha una complessità di $O(\log m)$. Il ciclo principale viene eseguito $m$ volte.

Quindi, la complessità complessiva dell'algoritmo greedy per MST è $O(n + m \log m)$.

È importante notare che se il grafo viene rappresentato come una matrice di adiacenza invece di una lista di adiacenza, la complessità può essere diversa, ad esempio $O(n^2)$ per la ricerca dell'arco di peso minimo.
\subsection{Complessit\'a dell'algoritmo}






\section{Algoritmo MST-1 - Prim}

\begin{center}
\begin{tikzpicture}[auto,node distance=2.5cm]
    \node (A) {A};
    \node (B) [below left of=A] {B};
    \node (C) [below right of=A] {C};
    \node (D) [below right of=B] {D};
    \node (E) [right of=C] {E};
    \node (F) [below right of=E] {F};
    \node (G) [below left of=F] {G};

    \path (A) edge node {2} (B)
              edge node {5} (C)
          (B) edge node {1} (D)
              edge node {3} (C)
          (C) edge node {2} (D)
              edge node {4} (E)
          (D) edge node {3} (F)
          (E) edge node {1} (F)
          (F) edge node {2} (G);
\end{tikzpicture}\end{center}


  \subsection{Procedimento}
  \begin{enumerate}
    \item Sceglie un nodo qualsiasi come nodo di partenza e lo aggiunge alla lista dei nodi visitati
    \item Scegliere un nodo non visitato che abbia un arco che lo colleghi a un nodo visitato con peso minimo e aggiungerlo alla lista dei nodi visitati
    \item Ripetere il passo 2 fino a quando non sono stati visitati tutti i nodi
  \end{enumerate}
  



\section{Algoritmo MST-2 - Kruskal}

\begin{center}
\begin{tikzpicture}[auto,node distance=2.5cm]
    \node (A) {A};
    \node (B) [below left of=A] {B};
    \node (C) [below right of=A] {C};
    \node (D) [below right of=B] {D};
    \node (E) [right of=C] {E};
    \node (F) [below right of=E] {F};
    \node (G) [below left of=F] {G};

    \path (A) edge node {2} (B)
              edge node {5} (C)
          (B) edge node {1} (D)
              edge node {3} (C)
          (C) edge node {2} (D)
              edge node {4} (E)
          (D) edge node {3} (F)
          (E) edge node {1} (F)
          (F) edge node {2} (G);
\end{tikzpicture}\end{center}


  \subsection{Procedimento}
  \begin{enumerate}
    \item Si ordinano tutti gli archi del grafo in ordine crescente del peso.
    \item Si aggiunge un arco alla volta alla lista degli archi dell'albero di supporto, a patto che non si creino cicli.
    \item Si ripete il passo 2 fino a che non si sono aggiunti $n-1$ archi, dove $n$ è il numero di nodi del grafo.
  \end{enumerate}








  \chapter{Algoritmi di cammini minimi}


  Ci sono due algotimi principali per risolvere il problema del cammino minimo:
  \begin{itemize}
    \item Algoritmo di Dijkstra: valido per grafi con pesi non negativi
    \item Algoritmo di Floyd-Warshall: valido per grafi con anche pesi negativi
  \end{itemize}


  \section{Algoritmo di Dijkstra}
  Percorso minimo da un nodo $s$ a tutti i nodi del grafo(grafo con pesi non negativi).

  \begin{enumerate}
    \item Si sceglie un nodo di partenza $s$ e si inizializza un vettore $d$ con la distanza di tutti i nodi da $s$.
    \item Si sceglie il nodo $v$ con distanza minima da $s$ e si aggiorna il vettore $d$ con le distanze dei nodi adiacenti a $v$.
    \item Si ripete il passo 2 fino a che non sono stati visitati tutti i nodi.
  \end{enumerate}


  \subsection{Complessit\`a}
  L'algoritmo di Dijkstra ha complessit\`a $O(n^2)$ se il grafo \`e rappresentato con una matrice di adiacenza, $O(m \log n)$ se il grafo \`e rappresentato con una lista di adiacenza.



  \section{Algoritmo di Floyd-Warshall}
  Percorso minimo tra tutti i nodi del grafo(grafo con pesi anche negativi).



\begin{algorithm}[H]
\SetAlgoLined
\KwIn{Matrice dei pesi $W$}
\KwOut{Matrice delle distanze minime $D$}

Inizializza la matrice delle distanze minime $D$ con i valori della matrice dei pesi $W$\;

\For{$k$ da $1$ a $n$}{
  \For{$i$ da $1$ a $n$}{
    \For{$j$ da $1$ a $n$}{
      $D[i][j] \leftarrow \min(D[i][j], D[i][k] + D[k][j])$\;
    }
  }
}

\Return{Matrice delle distanze minime $D$}\;
\caption{Algoritmo di Floyd-Warshall}
\end{algorithm}

L'algoritmo di Floyd-Warshall viene utilizzato per trovare i cammini minimi tra tutte le coppie di nodi in un grafo pesato, anche se ci sono archi con pesi negativi. L'algoritmo opera utilizzando una matrice delle distanze minime $D$, in cui $D[i][j]$ rappresenta la distanza minima tra il nodo $i$ e il nodo $j$. La matrice dei pesi $W$ contiene i pesi degli archi del grafo.

L'algoritmo procede iterativamente aggiornando i valori della matrice delle distanze minime $D$. In ogni iterazione, viene selezionato un nodo $k$ come possibile nodo intermedio nel cammino da $i$ a $j$. L'algoritmo confronta la distanza minima esistente tra $i$ e $j$ con la somma delle distanze minime tra $i$ e $k$ e tra $k$ e $j$. Se la somma delle distanze minime è inferiore alla distanza minima esistente, la distanza minima viene aggiornata.

L'algoritmo di Floyd-Warshall continua ad aggiornare le distanze minime per tutte le coppie di nodi utilizzando tutti i possibili nodi intermedi $k$. Alla fine delle iterazioni, la matrice delle distanze minime $D$ conterrà le distanze minime tra tutte le coppie di nodi nel grafo.

L'algoritmo di Floyd-Warshall ha una complessità di tempo di $O(n^3)$, dove $n$ è il numero di nodi nel grafo. Può essere utilizzato per risolvere problemi come la determinazione del cammino più breve tra due nodi o la rilevazione di cicli negativi nel grafo.

Spero che questa spiegazione dell'algoritmo di Floyd-Warshall sia stata utile! Se hai ulteriori domande, non esitare a chiedere.
