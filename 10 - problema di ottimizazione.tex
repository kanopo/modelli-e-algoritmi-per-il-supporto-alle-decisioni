\chapter{Problema di ottimizzazione}
Il problema di ottimizzazione è un tipo di problema matematico in cui si cerca di trovare il valore massimo o minimo di una funzione, nota come funzione obiettivo, in presenza di una serie di vincoli. Questi vincoli possono essere di natura diversa, ad esempio possono essere delle equazioni o delle disuguaglianze che limitano l'insieme di soluzioni ammissibili per il problema.

Il problema di ottimizzazione si presenta in molte applicazioni pratiche, come la programmazione lineare, la schedulazione, la gestione delle risorse, la progettazione di sistemi, l'ingegneria, l'economia e molti altri campi. Solitamente, la soluzione di questi problemi richiede l'utilizzo di algoritmi e tecniche matematiche avanzate, come la programmazione dinamica, la programmazione lineare, la teoria dei grafi, la teoria dei numeri e altre ancora.

\section{Problemi in P e classi NP}
Un problema di ottimizzazione è un problema in cui l'obiettivo è trovare la migliore soluzione possibile (in genere, la soluzione che massimizza o minimizza una certa funzione obiettivo). Ad esempio, il problema di trovare il percorso più breve in un grafo da un nodo sorgente a un nodo destinazione è un problema di ottimizzazione, in quanto l'obiettivo è trovare il percorso più breve.

La classe P (dall'inglese Polynomial time) è l'insieme dei problemi di decisione che possono essere risolti da un algoritmo in tempo polinomiale, cioè in un tempo che cresce al massimo come una potenza del grado del problema. Ad esempio, il problema di ordinare una lista di n numeri può essere risolto in tempo O(n log n), che è polinomiale.

La classe NP (dall'inglese Non-deterministic Polynomial time) è l'insieme dei problemi di decisione per i quali un'istanza positiva può essere verificata in tempo polinomiale da una macchina di Turing non deterministica. In altre parole, se esiste una soluzione per un problema in NP, allora può essere trovata in un tempo polinomiale. Tuttavia, non è noto se tutti i problemi in NP possano essere risolti in tempo polinomiale da una macchina di Turing deterministica.

Molti problemi pratici di ottimizzazione, come il problema del commesso viaggiatore e il problema dello zaino, sono NP-completi, il che significa che sono tra i problemi più difficili in NP. Sebbene non si conoscano algoritmi efficienti per risolverli, esistono algoritmi approssimati che forniscono soluzioni vicine all'ottimo in un tempo ragionevole.
