\chapter{Assegnamento}
L'assegnamento è un problema di ottimizzazione che consiste nel trovare la soluzione ottima di un insieme di assegnamenti tra un gruppo di lavoratori e un gruppo di mansioni, considerando i costi associati ad ogni possibile assegnamento. In altre parole, l'assegnamento cerca di minimizzare il costo totale dell'assegnamento dei lavoratori alle mansioni, soddisfacendo allo stesso tempo i vincoli di assegnamento.

Per questo tipo di task si utilizzano delle matrici che rappresentano i costi,
ecco un esempio di matrice dei costi per l'assegnamento di lavoratori a mansioni:

\begin{equation}
\begin{bmatrix}
 & A_1 & A_2 & A_3 & A_4 & A_5 \\
B_1 & c_{11} & c_{12} & c_{13} & c_{14} & c_{15} \\
B_2 & c_{21} & c_{22} & c_{23} & c_{24} & c_{25} \\
B_3 & c_{31} & c_{32} & c_{33} & c_{34} & c_{35} \\
B_4 & c_{41} & c_{42} & c_{43} & c_{44} & c_{45} \\
B_5 & c_{51} & c_{52} & c_{53} & c_{54} & c_{55}
\end{bmatrix}
\end{equation}

In questa matrice, $c_{ij}$ rappresenta il costo associato all'assegnamento del lavoratore $B_i$ alla mansione $A_j$.

\section{Algoritmo ungherese}
L'algoritmo ungherese, noto anche come algoritmo di Kuhn-Munkres, è un algoritmo per la soluzione del problema dell'assegnamento, ovvero il problema di assegnare N lavoratori a N mansioni, minimizzando i costi totali dell'assegnamento. L'algoritmo utilizza il concetto di cammino alternante e cammino aumentante per trovare l'assegnamento ottimale.
\subsection{Ridurre la matrice dei costi}

Per ridurre la matrice dei costi in un problema di assegnamento, puoi seguire i seguenti passi:
\begin{enumerate}
  
\item Sottrarre il minimo valore in ogni riga dalla riga stessa.
\item Sottrarre il minimo valore in ogni colonna dalla colonna stessa.
\item Coprire la matrice con il minor numero di linee orizzontali e verticali in modo che ogni cella sia coperta da almeno una linea.
\item Calcolare il costo minimo delle celle non coperte.
\item Sottrarre il costo minimo alle celle coperte dalle linee orizzontali e aggiungere il costo minimo alle celle coperte dalle linee verticali.
\item Ripetere i passi 3-5 finché non è possibile coprire la matrice con meno linee del passo precedente.

\end{enumerate}
La matrice ridotta avrà gli stessi elementi della matrice originale ma con alcune righe o colonne che contengono solamente zeri. 

Ecco un esempio di matrice dei costi ridotta:

$$
\begin{pmatrix}
0 & 1 & 5 & 8 & 1 \\
2 & 5 & 0 & 5 & 2 \\
1 & 0 & 3 & 6 & 7 \\
8 & 5 & 7 & 0 & 3 \\
3 & 3 & 2 & 1 & 0
\end{pmatrix}
\rightarrow
\begin{pmatrix}
0 & 0 & 4 & 7 & 0 \\
1 & 4 & 0 & 4 & 1 \\
0 & 0 & 2 & 5 & 6 \\
5 & 2 & 4 & 0 & 1 \\
2 & 2 & 1 & 0 & 0
\end{pmatrix}
\rightarrow
\begin{pmatrix}
x & 0 & 4 & 7 & 0 \\
0 & 4 & 0 & 4 & 1 \\
0 & 0 & 2 & 5 & 6 \\
5 & 2 & 4 & 0 & 1 \\
2 & 2 & 1 & 0 & 0
\end{pmatrix}
\rightarrow
\begin{pmatrix}
x & 0 & 4 & 7 & 0 \\
0 & 4 & 0 & 4 & 1 \\
0 & 0 & 2 & 5 & 6 \\
5 & 2 & 4 & 0 & 1 \\
0 & 0 & 0 & 0 & x
\end{pmatrix}
$$

Nella matrice ridotta, le righe 1, 2 e 4 e le colonne 2, 4 e 5 contengono solamente zeri. Il costo complessivo della soluzione ottima è pari alla somma dei valori nella matrice originale corrispondenti alle celle che contengono un "x".

\begin{algorithm}[H]
\SetAlgoLined
\KwResult{Assegnamento ottimale di lavoratori a mansioni}
Riduci la matrice dei costi;
Trova una soluzione di base ammissibile;
\While{non si raggiunge la soluzione ottimale}{
Trova un cammino aumentante;
Incrementa il flusso lungo il cammino aumentante;
Decrementa il flusso lungo gli archi corrispondenti nel cammino aumentante;
}
\caption{Algoritmo ungherese}
\end{algorithm}
